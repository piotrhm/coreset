\thispagestyle{empty}
\begin{abstract}
    W pracy przedstawimy stan wiedzy na temat budowania coresetów w kontekście algorytmu K-means.
    W szczególności poruszymy techniki takie jak geometryczna dekompozycja oraz losowe próbkowanie z \cite{DBLP:journals/ki/MunteanuS18}.

    Celem pracy jest przedstawienie wyników teoretycznych \cite{DBLP:journals/ki/MunteanuS18} jak i implementacja technik budowania coresetów, które mogą mieć zastosowanie praktyczne \cite{bachem2017scalable} \cite{10.1145/1007352.1007400}.
    Następnie zmierzymy dokładność konstrukcji korzystając z algorytmu Lloyd'a, a następnie porównamy wyniki względem obliczeń na całym zbiorze danych.
\end{abstract}
