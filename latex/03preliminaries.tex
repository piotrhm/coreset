\chapter{Notacja i niezbędne definicje}\label{preliminaries}

W tej pracy przyjmujemy, że działamy w przestrzeni euklidesowej w skończonym wymiarze $d > 0$.

\begin{definition}
    \emph{Norma typu $l_{2}$} dla $x \in \mathbb{R}^d$ to:
    \begin{equation}
        ||x||_{2} = ||x|| = \sqrt{ \sum_{i = 1}^{d} x_{i}^{2} }
    \end{equation}
\end{definition}

\noindent
Załóżmy, że mamy dany konkretny problem optymalizacyjny. 
Niech $F(x)$ oznacza zbiór dopuszczalnych rozwiązań tego problemu dla danych wejściowych $x$. 
Niech $c(y) \leq 0$ gdzie $ y\in F(x)$ będzie funkcją kosztu rozwiązania dla tego problemu.
Oznaczmy przez $c_{opt}(x)$ koszt rozwiązania optymalnego dla danych $x$.

\begin{definition}
    \emph{$\epsilon$-aproksymacja.} 
    Algorytm A nazywamy $\epsilon$-aproksymacyjnym, jeżeli dla dowolnych poprawnych danych wejściowych $x$, $A(x) \in F(x)$ oraz:
    \begin{equation}
        \frac{|c(A(x)) - C_{opt}(x) |}{ \max\{c(A(x)),C_{opt}(x) \} } \leq \epsilon
    \end{equation}
    gdzie $\epsilon \in [0,1]$ nazywamy \textit{błędem aproksymacji}.
\end{definition}

\begin{definition}
    \emph{Centroid} dla skończonego zbioru punktów $x_{1}, \dots, x_{k} \in \mathbb{R}^{d}$ to:
    \begin{equation}
        C = \frac{x_{1} + \dots + x_{k}}{k}
    \end{equation}
    gdzie punkt $C$ minimalizuje sumę kwardatów odległości pomiedzy nim samym a punktami $x_{1}, \dots, x_{k}$.
\end{definition}

\noindent
Często utożamaiamy centroid z punktami dla których jest on centroidem.
W takiej sytuacji użyjemy stwierdzenia \textit{środek centroidu} do nazwania zdefiniowanego w 2.3 faktycznego centroidu.

\begin{definition}
    \emph{Klaster} to skończony zbioru punktów $x_{1}, \dots, x_{k} \in \mathbb{R}^{d}$, które łaczy wspólna cech.
    W naszym kontekście wspólną cechą bedzie przyporządkowanie do tego samego centroidu.
\end{definition}

\section{K-means}

Zacznijmy od zdefiniowania problemu dla którego będziemy analizować konstrukcje coresetów.

\begin{definition}
    \emph{Problem K-means.} Niech $X$ to skończony zbiór punktów z $\mathbb{R}^{d}$. 
    Dla danego $X$ chcemy znalźć zbiór $k \in \mathbb{N}$ punktów $Q \subset \mathbb{R}^{d}$, który minimalizuje funkcję $\phi_{X}(Q)$ zdefiniowaną następująco:
    \begin{equation}
        \phi_{X}(Q) = \sum_{x \in X} d(x, Q)^{2} = \sum_{x \in X} \min_{q \in Q} || x - q ||^{2} 
    \end{equation}
\end{definition}

\noindent
Definicja 2.2 zakłada, że działamy w przestrzeni euklidesowej.
Uogólnioną wersję można zdefiniować analogicznie, zamieniając $d$ na odpowiednią funkcję miary w danej przestrzeni.

\begin{definition}
    \emph{Problem K-means - wersja ważona.} Niech $C$ to skończony zbiór punktów z $\mathbb{R}^{d}$ oraz niech $w$ będzie funkcją $C \rightarrow \mathbb{R}_{\ge0}$. 
    Dla danego $C$ oraz funkcji $w$ chcemy znaleźć zbiór $k \in \mathbb{N}$ punktów $Q$, który minimalizuje funkcję $\phi_{C}(Q)$ zdefiniowaną następująco:
    \begin{equation}
        \phi_{C}(Q) = \sum_{c \in C} w(c) d(c, Q)^{2}
    \end{equation}
\end{definition}

\noindent
Ewaluację funkcji $\phi$ dla optymalnego rozwiązania oznaczamy $\phi_{opt}^{k}(X)$ lub $OPT$.
W pracy często będziemy korzystać ze stwierdzenia \textit{optymalne rozwiązanie}, które oznacza $k$ elementowy zbiór dla którego wartość funkcji $\phi$ jest zminimalizowana. 
Funkcję $\phi$ w literaturze nazywamy błędem kwantyzacji.

\section{Coreset}

To jak definujemy coreset ściśle zależy od problemu, który optymalizujemy.
Zacznijmy od podstawowej definicji coresetu dla problemu K-means.

\begin{definition}
    \emph{Coreset.} Niech $X$ to skończony zbiór punktów z $\mathbb{R}^{d}$ oraz niech $Q$ to dowolny podzbiór $X$ rozmiaru co najwyżej $k$. 
    Skończony zbiór $C \subset R^{d}$ nazywamy $(\epsilon, k)$ coresetem, gdzie $\epsilon \in (0, 1)$, jeżeli zachodzi:
    \begin{equation}
        |\phi_{X}(Q) - \phi_{C}(Q)| \leq \epsilon\phi_{X}(Q)
    \end{equation}
\end{definition}

\noindent
Zauważmy, że taka definicja daje nam bardzo moce gwarancje teoretyczne.
Wartość funkcji $\phi_{C}(Q)$ aproksymuje $\phi_{X}(Q)$ z błędem $(1+\epsilon)$ dla dowolnego zbioru $k$ kandydatów na rozwiązanie $Q$.
Jest to na tyle istotne, że w literaturze odróżnimy taką wersję nazywając ją \textit{strong coresetem}.

\begin{definition}
    \emph{Coreset - weak.} Niech $X$ to zbiór punktów z $\mathbb{R}^{d}$ oraz $\epsilon \in (0, 1)$.
    Słabym coresetem nazywamy skończony zbiór $C \subset \mathbb{R}^d$ dla którego zachodzi:
    \begin{equation}
        |\phi_{X}(Q) - \phi_{C}(Q)| \leq \epsilon\phi_{X}(Q)
    \end{equation}
\end{definition}

%dyskretna/ciągła 