\chapter{Geometryczna Dekompozycja}\label{geo}
W tej części pracy przedstawimy konstrukcję budowy coresetu bazującą na geometrycznej dekompozycji problemu.
Punktem wyjścia były badania \cite{DBLP:journals/ki/MunteanuS18}, na których bazuje opisany w podrozdziale 4.4 algorytm.
Zgodnie z nimi budowa coresetu w kontekście problemu K-means to wieloetapowy proces, który jest sekwencją algorytmów z prac: \cite{Gonzalez1985ClusteringTM} \cite{10.1145/1007352.1007400} \cite{Arya2004LocalSH} \cite{DBLP:journals/ki/MunteanuS18}.
W rozdziale przyjmujemy następujący porządek analizy konstrukcji:
\begin{itemize}
    \item W sekcjach 4.1, 4.2, 4.3 opiszemy konstrukcje pomocnicze pochodzące z prac: \cite{Gonzalez1985ClusteringTM}, \cite{10.1145/1007352.1007400}, \cite{Arya2004LocalSH}.
    \item W sekcji 4.4 opiszemy właściwy algorytm z pracy \cite{DBLP:journals/ki/MunteanuS18}.
\end{itemize}

\section{Algorytm Gonzalez'a}

Pierwszy algorytm z którego opiszę to \textit{Farthest point algorithm} z pracy \cite{Gonzalez1985ClusteringTM}.
Jest to pierwszy algorytm aproksymacyjny rozwiązująych problem k-centrów z błędem nie większem niż $2$\textit{OPT}, gdzie \textit{OPT} to optymalne rozwiązanie.
Jego złożoność to $O(nk)$, gdzie $n$ to liczba punktów. 
\\~\\
Niech $S$ to bedzie zbiór który chcemy sklastrować oraz niech $T$ będzie podzbiorem $S$.
Zakładamy, że $|S| > k$ ponieważ w przeciwnym przypadku problem jest trywialnie rozwiązywalny.
Zbiór $T$ nazywamy $(k+1)$ kliką wysokości $h$ jeżeli moc zbioru $T$ jest równa $k+1$ oraz odległość pomiędzy parą dwóch rożnych punktów jest równa co najmniej $h$. 
Niech $OPT(S)$ oznacza optymalne rozwiązanie problemu k-centrów dla zbioru $S$.
Udowodnię teraz następujący lemat, którego dowód jest opisem \textit{Farthest point algorithm}.

\begin{lemma}
    Jeżeli w zbiorze $S$ istnieje $(k+1)$ klika wysokości $h$, to $OPT(S) \geq h$.
\end{lemma}

\begin{proof}
    
\end{proof}

\begin{algorithm}
    \caption{}
\begin{algorithmic}
    \State For each point not in $T$, the algorithm keeps $neighbor(p)$, the nearest point in $T$, and $dist(p)$, the distance from $p$ to $neighbor(p)$.
    \Procedure{Farthest point algorithm}{}
        \State $T \leftarrow \emptyset$
        \State $dist(p) \leftarrow \inf$ for all $p \in S$
        \While{$|T| \leq k$}                    
            \State $D \leftarrow max\{dist(p) | p \in S-T\}$
            \State choose $p^{'}$ from $S-T$ such $dist(p^{'}) = D$
            \State add $p^{'}$ to T
            \State update $neighbor(p)$ and $dist(p)$ for all $p \in S-T$
        \EndWhile
    \EndProcedure
    \Return $T$
\end{algorithmic}
\end{algorithm}
\section{Konstrukcja kraty wykładniczej}

W tym podrozdziale omówimy część pracy \cite{10.1145/1007352.1007400}.
Tematem pracy są konstrukcje algorytmów dla problemów $k$-means oraz $k$-median.
W kontekście problemu $k$-means autorzy zaproponowali algorytm rozwiązujący problem $k$-means bazujący na budowaniu coresetu, który uzyskuje lepszą złożoność od \cite{Matousek99onapproximate}.
Schemat konstrukcji jest następujący:
\begin{itemize}
    \item Obliczmy szybką ale niedokładną aproksymację dla problemu $k$-means z pewną dużą wartością $k$.
    \item Obliczoną aproksymację przekształcamy w $(\epsilon, k)$ coreset używając kraty wykładniczej.
\end{itemize}


\subsection{Szybka aproksymacja dla problemu K-means.}
\noindent
Zacznijmy od pierwszej części.
Dokładniej, udowodnimy następujące twierdzenie.

\begin{thm}{\cite{10.1145/1007352.1007400}}
    Dla danego zbioru $n$ punktów $P \subset \mathbb{R}^d$ oraz parametru $k \in \mathbb{N}$ możemy 
    obliczyć zbiór $X$ o mocy $O(k \log^{3}n)$, dla którego 
    \begin{equation}
        \phi_{P}(X) \leq 32 \phi_{opt}^{k}(P)
    \end{equation}
    Czas działania algorytmu to $O(n)$ dla $k = O(n^{\frac{1}{4}})$ oraz $O(n \log (k \log n))$ w przeciwnym przypadku.
\end{thm}

\noindent
Niech $P \subset \mathbb{R}^{d}$ będzie danym na wejściu zbiórem $k$ punktów.
Chcemy szybko obliczyć aproksymację dla problemu K-means na tym zbiorze, gdzie $k$ będzie rzędu $O(k \log^{3}n)$.

\begin{definition}
    \emph{Złe/dobre punkty.} Dla zbioru punktów $X$, punkt $p \in P$ nazywamy \textit{złym} jeżeli
    \begin{equation}
        d(p, X) \geq 2d(p, C_{opt})
    \end{equation}
    gdzie $C_{opt}$ jest zbiórem punktów realizującym $\phi_{opt}^{k}(P)$.
    Punkt jest \textit{dobry} jeżeli nie jest zły.
\end{definition}

\noindent
Na początku opiszemy procedurę, która dla danego $P$, wyznacza zbiór punktów $X$ oraz zbiór $P^{'} \subset P$.
Zbiór $P^{'}$ będzie zawierać \textit{dobre} punkty dla zbioru $X$.
\\~\\
Konstrukcję zbioru $X$ zaczynamy od obliczenia 2-aproksymacji problemu k-centrów dla zbioru $P$.
Niech obliczony zbiór centrów to $V$.
Taką aproksymację dla $k = O(n^{\frac{1}{4}})$ możemy obliczyć w czasie $O(n)$ oraz dla $k = \Omega(n^{\frac{1}{4}})$ w czasie $O(n \log k)$ \cite{10.1145/62212.62255}.
Niech $L$ będzie promieniem takiej aproksymacji, czyli największą odległością pomiędzy centrem a punktem $p \in P-V$.
Ponieważ \cite{10.1145/62212.62255} bazuje na \cite{Gonzalez1985ClusteringTM} to dystans pomiędzy dowolną parą punktów z $V$ to conajmniej $L$.
To implikuje następujące ograniczenia:
\begin{equation}
    \Big( \frac{L}{2 } \Big)^2 \leq \phi_{opt}^{k}(V) \leq \phi_{opt}^{k}(P) \leq nL^{2}
\end{equation} 
\\~\\
Następnym krokiem konstrukcji jest wylosowanie $\rho = \gamma k \log^{2} n$ punktów ze zbioru $P$.
Niech wylosowane punkty to $Y$, gdzie $\gamma$ jest stałą, która wynika z analizy, którą zaraz przeprowadzimy.
Finalnie, $X = Y \cup V$ będzie zbiorem centrów klastów.
Dla $\rho > n$ przyjmujemy $X = P$.
\\~\\
Konstrukcja zbioru $X$ jest stosunkowo prosta.
Dużo cięższym zadaniem jest zbudowanie zbioru $P^{'}$, który jest zbiorem \textit{dobrych} punktów dla $X$.

\subsection{Konstrukcja zbioru dobrych punktów dla $X$.}

Rozpatrzmy zbiór $C_{opt}$, który jest optymalnym zbiorem centrów dla problemu K-means w kontekście $P$.
Dla każdego $c_{i} \in C_{opt}$ tworzymy kulę $b_{i}$ o środku w $c_{i}$.
Każda taka kula będzie zawierać $\eta = \frac{n}{20k \log n}$ punktów z $P$.
Jeżeli $\gamma$ jest odpowiednio duże to z wysokim prawdopodobieństwem w każdym $b_{i}$ jest przynajmniej jeden punkt z $X$.
Dokładniej:
\begin{equation}
    X \cap b_{i} \neq \emptyset \text{ dla } i = 1 \dots k
\end{equation}

\noindent
Niech $P_{bad}$ będzie zbiorem złych punktów $P$ w kontekście zbioru $X$.
Załóżmy, że dla każdego $b_{i}$ istnieje punkt $x_{j} \in X$, który $x_{j} \in b_{i}$.
Zauważmy, że dla każdego punktu $p \in P \setminus b_{i}$, dla którego $x_{j}$ jest centerm mamy $||px_{j}|| \leq ||pc_{i}||$.
W szczególności takie punkty są \textit{dobre}, dla $c_{i}$ będącymi optymalnymi centrami dla tych punktów.
Zatem z wysokim prawdopodobieństwem jedyne \textit{złe} punkty będą w kulach $b_{i}$ dla $ i = 1, \dots, k$.
To implikuje, że z wysokim prawdopodobieństwem liczba złych punktów w $P$ dla zbioru $X$ to co najwyżej $\beta = k\eta = \frac{n}{20 \log n}$.
\\~\\
W takim razie złych punktów nie jest dużo.
Mimo tego bezpośrednie wyznaczenie tych punktów jest skomplikowane.
Autorzy pracy \cite{10.1145/1007352.1007400} budują zbiór $P^{'}$ tak aby koszt złych punktów w $P^{'}$ był jak najmniejszy.
Koszt w tym kontekście oznacza to jaki wkład ma punkt w wartość $\phi_{P^{'}}(X)$.
Dla każdego punktu w $P$ obliczamy najbliższego sąsiada w $X$.
Niech $r(p) = d(p, X)$ dla każdego punktu $p \in P$.
Teraz podzielimy P na zbiory według następującej formuły:
\begin{equation}
    P[a,b] = \{ p \in P \text{ | } a \leq r(p) \leq b \}
\end{equation}
A dokładniej:
\begin{equation}
    P_{0} = P\Big[0, \frac{L}{4n}\Big]
\end{equation}
\begin{equation}
    P_{ \infty } = P\Big[2Ln, \infty \Big]
\end{equation}
\begin{equation}
    P_{i} = P\Big[ \frac{2^{i-1}L}{n}, \frac{2^{i}L}{n} \Big]
\end{equation}
\\~\\
dla $i = 1, \dots, M$, gdzie $M = 2 \lceil \lg n \rceil + 3$.
Taki podział możemy wykonać w czasie linowym.
Niech $P_{\alpha}$ będzię ostatnim zbiorem, który zawiera więcej niż $2\beta = \frac{n}{10 \log n}$. 
Szukany zbiór zdefiniujemy następująco:
\begin{equation}
    P^{'} = V \cup \bigcup_{i \leq \alpha} P_{i}
\end{equation}
gdzie $|P^{'}| \geq \frac{n}{2}$ oraz $\phi_{P^{'}}(X) = O(\phi_{P^{'}}(C_{opt}))$.
Teraz udowodnimy, że faktycznie tak zdefiniowane $P^{'}$ spełnia powyższe założenia.

\begin{proof}
    Moc zbioru $P^{'}$ jest na pewno równa conajmniej $\Big(n - |P_{\infty}| - M\frac{n}{10 \log n} \Big)$.
    Zauważmy, że $P_{\infty} \subseteq P_{bad}$ oraz $|P_{bad}| \leq \beta$.
    A więc:
    \begin{equation}
        |P^{'}| \geq n - \frac{n}{20 \log n} - M \frac{n}{10 \log n}
    \end{equation}
    \begin{equation}
        = n - \Big(\frac{n}{10 \log n}\Big) \Big(M + \frac{1}{2}\Big)
    \end{equation}
    \begin{equation}
        = n - \Big(\frac{n}{10 \log n}\Big) \Big(2 \lceil \lg n \rceil + 3 + \frac{1}{2}\Big)
    \end{equation}
    \begin{equation}
        \geq \frac{n}{2}
    \end{equation}
    Jeżeli $\alpha > 0$, to $|P_{\alpha}| \geq 2\beta = \frac{n}{10 \log n}$.
    Teraz chcemy oszacować jaką kontrybucję w $P^{'}$ mają złe punkty.
    Z uwagi na to jak budujemy $P^{'}$ w najgorszym przypadku wszystkie złe punkty będą w $P_{\alpha}$.
    Wtedy takie punkty będą miały największy wpływ na funkcję $\phi$.
    Niech $Q^{'} = P_{\alpha} \setminus P_{bad}$.
    Dla dowolnego punktu z $p \in P^{'} \cap P_{bad}$ oraz $q \in Q^{'}$, mamy $d(p, X) \leq 2d(q,X)$.
    Dodatkowo $|Q^{'}| > |P_{bad}|$, a więc:
    \begin{equation}
        \phi_{P^{'} \cap P_{bad}}(X) \leq 4\phi_{Q^{'}}(X) \leq 16\phi_{Q^{'}}(C_{opt}) \leq 16\phi_{P^{'}}(C_{opt})
    \end{equation}
    Teraz możemy wyprowadzić następujące ograniczenie:
    \begin{equation}
        \phi_{P^{'}}(X) = \phi_{P^{'} \cap P_{bad}}(X) + \phi_{P^{'} \setminus P_{bad}}(X)
    \end{equation}
    \begin{equation}
        \leq 16\phi_{P^{'}}(C_{opt}) + 4\phi_{P^{'}}(C_{opt}) = 20\phi_{P^{'}}(C_{opt})
    \end{equation}
    Jeżeli $\alpha = 0$, to dla dowolnego punktu $p \in P^{'}$ mamy:
    \begin{equation}
        (d(p,X))^2 \leq n\Big(\frac{L}{4n}\Big)^2 \leq \frac{L^{2}}{4n}
    \end{equation}
    Zatem:
    \begin{equation}
        \phi_{P^{'}}(X) \leq \frac{L^{2}}{4} \leq \phi_{V}(C_{opt}) \leq \phi_{P^{'}}(C_{opt})
    \end{equation}
    gdzie $V \subseteq P^{'}$.
\end{proof}

\noindent
Powyższa analiza dowodzi poprawności konstrukcji dla zbiorów $X$ i $P^{'}$.
Podsumowując otrzymujemy zbiór $X$ o mocy $O(k \log^{2} n)$ oraz zbiór $P^{'}$, dla którego mamy $\phi_{P^{'}}(X) \leq 32\phi_{P^{'}}(C_{opt})$.
Czas działania tego algorytmu to $O(n)$ dla $k = O(n^{\frac{1}{4}})$ oraz $O(n \log (k \log n))$ w przeciwnym przypadku.
Aby otrzymać taką złożoność kluczowe jest szybkie oblicznie najbliższych sąsiadów.
Autorzy proponują \cite{10.1145/293347.293348}.
\\~\\
Wróćmy teraz do twierdzenia 4.1, które zostało zdefiniowane na początku podrozdziału 4.2.
Powyżej zdefiniowaną procedurę powtarzamy dla zbioru $P_{1} = P \setminus P^{'}$.
Analogicznie otrzymamy zbiór $P_{1}^{'}$ oraz $X_{1}$.
Kolejny raz aplikujemy procedurę na zbiorze $P_{2} = P \setminus (P^{'} \cup P_{1}^{'})$.
Ponieważ za każdym razem zbiór $P_{i}$ zmieniejsza się o połowę to taki proces zakończy się po $O(\log n)$ powtórzeniach.
Finalnie otrzymamy zbiór $X \cup X_{1} \dots X_{i}$ o mocy $O(k \log^{3} n)$, dla którego $\phi_{P}(X) \leq 32\phi_{P}(C_{opt})$.
Złożoność pozostanie taka sama, czyli $O(n)$ dla $k = O(n^{\frac{1}{4}})$ oraz $O(n \log (k \log n))$ w przeciwnym przypadku.

\subsection{Krata wykładnicza.}

Przejdzmy teraz do kluczowej części tego podrozdziału, czyli konstrukcji kraty wykładniczej.
Niech $P \subset \mathbb{R}^d$, $|P| = n$ oraz niech $A = \{x_{1}, \dots, x_{m}\}$ będzie zbiorem punktów, dla którego zachodzi $\phi_{P}(A) \leq c\phi_{opt}^{k}(P)$, gdzie $c$ jest stałą.
Nasze $A$ otrzymamy z konstrukcji opisanej w 4.2.1 dla $k = O(k poly \log n)$, gdzie $poly \log n = \bigcup_{c \geq 1} O(\log^{c}n)$.
\\~\\
Niech $R = \sqrt{\frac{\phi_{P}(A)}{cn}}$ będzie dolnym ograniczeniem dla $R_{opt}^{\phi}(P, k) = \sqrt{\frac{\phi_{opt}(P, k)}{n}}$
Niech $P_{i}$ będzie podzbiorem punktów z $P$, dla których punkt $x_{i} \in A$ jest dla niech najbliższym sąsiadem. 
Dla dowolnego $p \in P_{i}$, mamy $||px_{i}|| \leq \sqrt{xn}R$, ponieważ $||px_{i}||^{2} \leq \phi_{P}(A)$ dla $i = 1, \dots, m$.
\begin{figure}[H]
    \centering
    \includegraphics[totalheight=4cm]{grid.png}
    \caption{Krata wykładnicza, gdzie $c = x_{i}$.}
\end{figure}
\noindent
Kolejnym krokiem będzie budowa kraty wykładniczej wokół każdego punktu $x_{i}$ oraz nałożenie jej na zbiór $P$.
Niech $Q_{i,j}$ będzie kwatratem o boku długości $R2^{j}$ o środku w punkcie $x_{i}$ dla $j = 0, \dots, M$, gdzie $M = \lceil 2 \lg(cn) \rceil$, który jest równoległy do osi układu współrzednych dla danej przestrzeni.
Następnie, niech $V_{i, 0} = Q_{i, 0}$ oraz niech $V_{i,j} = Q_{i,j} \setminus Q_{i,j}$ dla $j = 0, \dots, M$.
Kolejnym krokiem będzie przekształcenie $V_{i,j}$ w kratę, której komórki będą długości $r_{j} = \frac{\epsilon R2^{j}}{10cd}$ oraz niech $G_{i}$ oznacza wynikową kratę wykładniczą dla $V_{i,0}, \dots, V_{i,M}$.
Mając $G_{i}$ obliczamy dla każdego punktu z $P_{i}$ komórkę, do której należy.
Dla każdej niepustej komórki z kraty wybieramy losowy punkt z $P_{i}$, który będzie jej reprezentantem.
Do takiego punktu przypisujemy wagę, która bedzie równa liczbie punktów z komórki, dla której jest reprezentantem.
Po przejściu całej kraty otrzymamy zbiór $S_{i}$ takich punktów.
Definujemy $S = \bigcup_{i} S_{i}$, który jest  $(\epsilon, k)$ coresetem.
Zauważmy, że $|S| = O\Big( \frac{|A| \log n }{ (c\epsilon)^{d} } \Big)$, ponieważ każda krata ma $\log n$ poziomów, a każdy poziom stałą liczbę komórek.


\section{Heurystyka single swap}

W tym podrozdziale opiszemy heurystykę single swap \cite{Arya2004LocalSH}, która jest przykładem techinki \textit{local search}.
Algorytm jest $(25 + \epsilon)$ aproksymacją problemu $k$-means, która zakłada, że na wejściu dostajemy zbiór kandydatów na centra $C$ oraz zbiór $n$ punktów $P \subset \mathbb{R}^d$.
Autorzy \cite{Arya2004LocalSH} w celu wyznaczenia $C$ wykorzystują algorytm z pracy \cite{Matousek99onapproximate}, którą zastąpimy pracą \cite{10.1145/1007352.1007400} z uwagi na lepsze gwarancje teoretyczne.
Korzystając z algorytmu opisanego w podrozdziale 4.2.3 wyznaczamy zbiór $C$, który będzie obliczonym $(\epsilon, k)$-coresetem.
\\~\\
Heurystyka \textit{single swap} działa poprzez wybranie początkowego zestawu $k$ centrów $S$ ze zbioru kandydatów na centra $C$, a następnie wielokrotnej
próbie ulepszenia rozwiązania poprzez usunięcie jednego centrum $s \in S$ i zastąpienie go innym centerm $s^{'} \in C - S$.
Początkowy stan $S$ może zostać zainicjalizowany losowo albo może być obliczony za pomocą omówionego w podrozdziale 4.1 algorytmu Gonzaleza.
Niech $S^{'} = S - \{s\} \cup \{s^{'}\}$ będzie nowym zbiorem centrów, gdzie taką operację na zbiorze $S$ nazwyamy \textit{wymianą} a punkty $s$, $s^{'}$ \textit{swap parą}.
Jeżeli $\phi_{P}(S^{'}) \leq \phi_{P}(S)$ to zastępujemy zbiór $S$ zbiorem $S^{'}$, w przeciwnym przypadku $S$ pozostaje bez zmian.
W praktyce taki proces powtarzamy do momentu, kiedy $|\phi_{P}(S^{'}) - \phi_{P}(S) | < \epsilon$.
Formalnie możemy udowodnić, że dla każdego $\epsilon > 0$, po wielomianiwej liczbie wymian punktów $s$, $s^{'}$ algorytm zakończy swoje działanie.
Autorzy nie dowodzą tego wprost ale powołują się na pracę \cite{10.1145/380752.380755}.
\\~\\
Dla uproszczenia zakładamy, że algorytm kończy się kiedy pojedyńcza wymiana elementów $s$, $s^{'}$ nie poprawia wyniku.
Taki zbiór centrów nazwiemy \textit{1-stable}.
\\~\\
Wprowadźmy notację:
\begin{equation}
    \Delta(u, v) = d(u, v)^{2} = \sum_{i=0}^{d} (u_{i} - v_{i})^2 = (u - v)\cdot(u - v)
\end{equation}
gdzie punkty interpretujemy jako wektory oraz operacja $(\cdot)$ jest iloczynem skalarnym.
\begin{equation}
    \Delta(S, v) = \sum_{u \in S} d(u, v)^{2}
\end{equation}
\begin{equation}
    \phi_{P}(S) = \Delta(S) = \sum_{q \in P} \Delta(q,s_{q}) = \sum_{q \in P} d(q, s_{q})^{2}
\end{equation}
gdzie $s_{q}$ to najbliższy punkt dla $q$ w $S$ oraz $P \subset \mathbb{R}^{d}$. 

\begin{definition}
    Zbiór $S$ nazywamy \emph{1-stable}, jeżeli:
    \begin{equation}
        \Delta(S - \{s\} \cup \{c\}) \leq \Delta(S)
    \end{equation}
    dla dowolnych $s \in S$ oraz $c \in C_{opt}$.
\end{definition}

\begin{definition}
    Niech $N_{S}(s)$ oznacza sąsiedzctwo punktu $s$, czyli zbiór punktów z $P$, dla których punkt $s$ jest najbliżej spośród punktów z $S$.
\end{definition}

\noindent
W ramach tego podrozdziału udowodnimy następujące twierdzenie.

\begin{thm}{\cite{10.1145/1007352.1007400}}
    Niech $S$ będzie zbiorem $k$ punktów spełniającym definicję 1-stable oraz niech $C_{opt}$ będzie optymalnym zbiorem dla problemu $k$-means.
    Wtedy zachodzi następującą nierówność:
    \begin{equation}
        \Delta(S) \leq 25 \Delta(C_{opt})
    \end{equation} 
\end{thm}

\begin{lemma}{\cite{10.1145/1007352.1007400}}
    Dla danego skończonego podzbioru $S$ punktów z $\mathbb{R}^d$, niech $c$ będzie centroidem dla $S$. Wtedy dla dowolnego $c^{'} \in \mathbb{R}^d$, $\Delta(S, c^{'}) = \Delta(S, c) + |S|\Delta(c, c^{'})$.
\end{lemma}
\begin{proof}
    Z definicji $\Delta(S, c^{'})$ otrzymujemy:
    \begin{equation}
        \Delta(S, c^{'}) = \sum_{u \in S} \Delta(u, c^{'}) =  \sum_{u \in S} (u - c^{'}) (u - c^{'})
    \end{equation}
    \begin{equation}
         = \sum_{u \in S} ((u - c) + (c - c^{'})) ((u - c) + (c - c^{'}))
    \end{equation}
    \begin{equation}
        = \sum_{u \in S} ((u - c)(u - c)) + 2((u - c)(c - c^{'})) + ((c - c^{'})(c - c^{'}))
    \end{equation}
    \begin{equation}
        = \Delta(S, c) + 2\Big( (c - c^{'}) \sum_{u \in S} (u - c) \Big) + |S|((c - c^{'})(c - c^{'}))
    \end{equation}
    \begin{equation}
        = \Delta(S, c) + |S|\Delta(c,c^{'})
    \end{equation}
    Ostanie przejście korzysta z faktu, że jeżeli $c$ jest centroidem $S$ to z definicji zachodzi $\sum_{u \in S} (u - c) = 0$.
\end{proof}

%\epsilon
\noindent
Na potrzebę dowodu załóżmy, że znamy zbiór $C_{opt}$.
Dla każdego optymalnego $c \in C_{opt}$ wyznaczamy $s_{c}$, który jest najbliższym punktem w $S$ dla punktu $c$.
Punkt $s_{c}$ nazywamy heurystycznym centrum dla punktu $c$.
W takim kontekście powiemy, że $c$ jest \textit{schwytane} przez $s_{c}$.
Tutaj warto zaznaczyć, że każde optymalne centrum jest schwytane przez jedno heurystyczne centrum, ale każde heurystyczne centrum może być schwytane przez kilka optymalnych centrów.
Heurystyczne centrum nazwiemy \textit{samotnym} jeżeli nie jest schwytane przez żadne optymalne centrum.

\begin{proof}
    Dowód twierdzenia 4.2 zaczniemy od zdefniowania podziału $S$ oraz $C_{opt}$ na zbiory $S_{1}, \dots, S_{r}$ oraz $O_{1}, \dots, O_{r}$ dla pewnego $r$, gdzie $|S_{i}| = |O_{i}|$ dla $i = 1, \dots, r$.
    \\~\\
    Dla każdego heurystycznego centrum $s_{i}$, które schywało jakąś liczbę $m \geq 1$ optymalnych centrów, tworzymy zbior $S_{i}$, który będzie zawierał $s_{i}$ oraz dowolne $m-1$ osamotnionych heurystycznych centrów.
    Analogicznie, zbior $O_{i}$ będzie zawierał wszystkie optymalne centra schwytane przez $s_{i}$.
    Rysunek 4.2 obrazuje tak zdefiniowany podział.
    \\~\\
    \noindent
    Wygenerujemy teraz \textit{swap par} dla utworzonego podziału.
    Dla każdego podzbioru podziału takiego, że $|S_{i}| = |O_{i}| = 1$ , tworzymy z ich elementów parę.
    Dla każdego podzbioru podziału, który zawiera więcej schwytanych centrów, czyli $|S_{i}|, |O_{i}| \geq 1$, tworzymy pary między osamotnionymi heurystycznymi centrami z $S_{i}$ a optymalnymi centrami z $O_{i}$.
    Każde optymalne centrum jest związane z jednym heurystycznym oraz każde osamotnione centrum jest przyporządkowane co najwyżej dwóm optymalnym centrom.
    Centra łaczymy dowolnie. 
    Rysunek 4.2 przedstawia przykładowe swap pary.
    \begin{figure}[H]
        \centering
        \includegraphics[totalheight=8cm]{swap.png}
        \caption{}
    \end{figure}
    Wyznaczymy teraz ograniczenie górne na zmianię funkcji $\Delta$ po wymianie punktów ze swap pary $(s, o)$.
    Zaczniemy od obliczenia najbliższych centrów z $S - \{s\} \cup \{o\}$ dla zadanego na wejściu zbioru $P$.
    Niech $N_{X}(x)$ będzie \textit{sąsiedztwem} punktu $x$, czyli podzbiorem punktów z $P$, dla których $x$ jest najbliższym punktem spośród punktów z $X$. 
    Dla punktów, które należą do $N_{C_{opt}}(o)$ zmiana $\Delta$ będzie następująca:
    \begin{equation}
        \sum_{q \in N_{C_{opt}}(o)} (\Delta(q, o) - \Delta(q, s_{q}))
    \end{equation}
    Każde $q \in N_{S}(s) \setminus N_{C_{opt}}(o)$ straciło przypisane mu centrum $s$ zatem punkt $q$ musi otrzymać nowe centrum.
    Niech $o_{q}$ będzie oznaczało najbliższe centrum dla punktu $q$.
    Skoro $q \notin N_{C_{opt}}(o)$ to $o_{q} \neq o$, zatem $s$ nie schwytało $o_{q}$.
    Zatem po skorzystaniu ze swap pary $(s,o)$, $s_{o_{q}}$, najbliższe heurystyczne centrum dla $o_{q}$ nadal istnieje.
    Zmiana $\Delta$ po wyborze nowych centrów jest co najwyżej równa:
    \begin{equation}
        \sum_{q \in N_{S}(s_{i}) \setminus N_{C_{opt}}(o_{i})} (\Delta(q, s_{o_{q}}) - \Delta(q, s))
    \end{equation}
    Na tym etapie dowodu koniecznie będzie wprowadzenie dwóch lematów.
    
    \begin{lemma}{\cite{10.1145/1007352.1007400}}
        Niech $S$ będzie zbiorem $k$ punktów spełniającym definicję 1-stable oraz niech $C_{opt}$ będzie optymalnym zbiorem dla problemu $k$-means.        Wtedy zachodzi następującą nierówność:
        \begin{equation}
            0 \leq \Delta(C_{opt}) - 3\Delta(S) + 2R
        \end{equation}
        gdzie, $R = \sum_{q \in P} \Delta(q, s_{o_{q}})$.
    \end{lemma}
    \begin{proof}
        Na potrzeby dowodu ustalmy swap parę $(s, o)$.
        Ponieważ $S$ jest 1-stable to:
        \begin{equation}
           0 \leq \Delta(S) -  \Delta(S - \{s\} \cup \{o\})
        \end{equation}
        \begin{equation}
            = \sum_{q \in N_{C_{opt}}(o)} (\Delta(q, o) - \Delta(q, s_{q})) + \sum_{q \in N_{S}(s) \setminus N_{C_{opt}}(o)} (\Delta(q, s_{o_{q}}) - \Delta(q, s))
        \end{equation}
        Aby rozszerzyć sumę na wszystkie możliwe swap pary zauważmy, że dla każdego optymalnego centrum, jest ono wymienione tylko raz.
        Zatem każdy punkt $q$ kontrybuuje w pierwszej sumie tylko raz.
        Po drugie zaważmy, że różnca w drugiej sumie jest zawsze niezerowa dlatego rozszerzając zakres sumowania mozemy tylko zwiększyć sumaryczny wynik.
        \begin{equation}
            0 \leq \sum_{q \in P} (\Delta(q, o_{q}) - \Delta(q, s_{q})) + \sum_{q \in P} (\Delta(q, s_{o_{q}}) - \Delta(q, s_{q}))
        \end{equation}
        \begin{equation}
            0 \leq \sum_{q \in P} \Delta(q, o_{q}) - 3 \sum_{q \in P} \Delta(q, s_{q}) + 2\sum_{q \in P} \Delta(q, s_{o_{q}})
        \end{equation}
        \begin{equation}
            0 \leq \Delta(C_{opt}) - 3 \Delta(S) + 2R
        \end{equation}
    \end{proof}
    \noindent
    Wcześniej zdefiniowane $R$ nazywamy sumarycznym kosztem przepisania centrów.
    Korzystając z lematu 3 przekształcamy:
    \begin{equation}
        R = \sum_{o \in C_{opt}} \sum_{q \in N_{C_{opt}}(o)} \Delta(q, s_{o}) = \sum_{o \in O} \Delta(N_{C_{opt}}(o), s_{o}) 
    \end{equation}
    \begin{equation}
        = \sum_{o \in C_{opt}} (\Delta(N_{C_{opt}}(o), o) + |N_{o}(C_{opt})| \Delta(o, s_{o})
    \end{equation}
    \begin{equation}
        = \sum_{o \in C_{opt}} \sum_{q \in N_{C_{opt}}(o)} (\Delta(q, o) + \Delta(o, s_{o}))
    \end{equation}
    \begin{equation}
        \leq \sum_{o \in C_{opt}} \sum_{q \in N_{C_{opt}}(o)} (\Delta(q, o) + \Delta(o, s_{q}))
    \end{equation}
    \begin{equation}
        = \sum_{q \in P} (\Delta(q, o_{q}) + \Delta(o_{q}, s_{q}))
    \end{equation}
    gdzie ostatnia nierówność bazuje na fakcie, że dla każdego $q \in N_{C_{opt}}(o)$ mamy $\Delta(o, s_{o}) \leq \Delta(o, s_{q})$.
    Następnie korzystamy z nierówności trójkąta.
    \begin{equation}
        R \leq  \sum_{q \in P} \Delta(q, o_{q}) + \sum_{q \in P} ( d(o_{q}, q) + d(q, s_{q}))^{2}
    \end{equation}
    \begin{equation}
        = \sum_{q \in P} \Delta(q, o_{q}) + \sum_{q \in P} ( d(o_{q}, q)^{2} + 2d(o_{q}, q)d(q, s_{q}) + d(q, s_{q})^{2})
    \end{equation}
    \begin{equation}
        = 2\sum_{q \in P} \Delta(q, o_{q}) + \sum_{q \in P} \Delta(q, s_{q}) + 2\sum_{q \in P} d(o_{q}, q)d(q, s_{q})
    \end{equation}
    \begin{equation}
        = 2\Delta(C_{opt}) + \Delta(S) + 2\sum_{q \in P} d(o_{q}, q)d(q, s_{q})
    \end{equation}
    \begin{lemma}{\cite{10.1145/1007352.1007400}}
        Niech $\langle o_{i} \rangle$ oraz $\langle s_{i} \rangle$ będą ciągami liczb rzeczywistych, dla których zachodzi:
        \begin{equation}
            \delta^2 = \frac{\sum_{i} s_{i}^{2}}{\sum_{i} o_{i}^{2}}
        \end{equation}
        dla pewnego $\delta > 0$.
        Wtedy:
        \begin{equation}
            \sum_{i=1}^{n} o_{i} s_{i} \leq \frac{1}{\delta} \sum_{i=1}^{n} s_{i}^{2}
        \end{equation}
    \end{lemma}
    \begin{proof}
        Z nierówności Schwarza:
        \begin{equation}
            \sum_{i=1}^{n} o_{i} s_{i} \leq \Big( \sum_{i=1}^{n} o_{i}^{2} \Big)^{\frac{1}{2}} \Big( \sum_{i=1}^{n} s_{i}^{2} \Big)^{\frac{1}{2}}
        \end{equation}
        \begin{equation}
            = \Big( \frac{1}{\delta^{2}}\sum_{i=1}^{n} s_{i}^{2} \Big)^{\frac{1}{2}} \Big( \sum_{i=1}^{n} s_{i}^{2} \Big)^{\frac{1}{2}}
        \end{equation}
        \begin{equation}
            = \frac{1}{\delta} \sum_{i=1}^{n} s_{i}^{2}
        \end{equation}
    \end{proof}
    Niech $\langle o_{i} \rangle$ będzie ciągiem $d(q, o_{q})$ oraz niech $\langle s_{i} \rangle$ będzie ciągiem $d(q,s_{q})$ dla wszystkich $q \in P$.
    Z tego wynika, że współczynnik aproksymacji możemy przedstawić jako:
    \begin{equation}
        \delta^{2} = \frac{\Delta(S)}{\Delta(C_{opt})} = \frac{\sum_{q \in P} d(q,s_{q})^{2}}{\sum_{q \in P} d(q,o_{q})^{2}} =\frac{\sum_{i=1}^{n} s_{i}^{2}}{\sum_{i=1}^{n} o_{i}^{2}}
    \end{equation}
    gdzie $S$ jest zbiorem 1-stable uzyskanym przez wcześniej przedstawiony algorytm.
    Korzystając z lematu 5:
    \begin{equation}
        R \leq 2\Delta(C_{opt}) + \Delta(S) + 2\sum_{q \in P} d(o_{q}, q)d(q, s_{q}) 
    \end{equation}
    \begin{equation}
        \leq 2\Delta(C_{opt}) + \Delta(S) + \frac{2}{\delta}\sum_{q \in P} d(q, s_{q})^{2} 
    \end{equation}
    \begin{equation}
        = 2\Delta(C_{opt}) + \Delta(S) + \frac{2}{\delta}\Delta(S)
    \end{equation}
    \begin{equation}
        = 2\Delta(C_{opt}) + \Big(1 + \frac{2}{\delta} \Big)\Delta(S)
    \end{equation}
    Z lematu 4 wiemy, że:
    \begin{equation}
        0 \leq \Delta(C_{opt}) - 3\Delta(S) + 2R
    \end{equation}
    \begin{equation}
        = \Delta(C_{opt}) - 3\Delta(S) + 2\Big(2\Delta(C_{opt}) + \Big(1 + \frac{2}{\delta} \Big)\Delta(S)\Big)
    \end{equation}
    \begin{equation}
        \leq 5\Delta(C_{opt}) - \Big(1 - \frac{4}{\delta} \Big)\Delta(S)
    \end{equation}
    Powyższą nierówności możemy przekształcić do postaci:
    \begin{equation}
        \frac{5}{1-\frac{4}{\delta}} \geq \frac{\Delta(S)}{\Delta(C_{opt})} = \delta^{2}
    \end{equation}
    \begin{equation}
        5 \geq \delta^{2} \Big(1 - \frac{4}{\delta} \Big)
    \end{equation}
    \begin{equation}
        0 \geq (\delta - 5)(\delta + 1)
    \end{equation}
    To implikuje, że $\delta \leq 5$, zatem współczynnik omawianego algorytmu możemy ograniczyć przez $\delta^{2} \leq 25$, co kończy dowód twierdzenia 4.2.
\end{proof}
\section{Podsumowanie}

W tej części przedstawimy algorytm budowania coresetu z \cite{DBLP:journals/ki/MunteanuS18}.
Algorytm ten rozpoczynamy od wykonania 10-aproksymacji dla problemu $k$-means korzystając z pracy \cite{Arya2004LocalSH}.
W części 4.3 opisaliśmy konstrukcję, której użyjemy w implementacji.
Ma ona trochę większy współczynnik aproksymacji ale jak sami autorzy \cite{DBLP:journals/ki/MunteanuS18} stwierdzają w swojej pracy, nie ma to dużego znaczenia w kontekście całości.
Dowolna aproksymacja o stałym błędzie może zostać użyta w tej metodzie.
Na potrzeby analizy zakładamy, że wykonujemy algorytm 10-aproksymacyjny dla problemu $k$-means i uzyskaliśmy zbiór $C^{'}$ oraz, że mamy dany na wejściu zbiór punktów $A \subset \mathbb{R}^d$.
\\~\\
Geometryczna dekompozycja bazuje na dyskretyzacji punktów z $A$, czyli na zgrupowaniu ze sobą najbliższych punktów, a następnie zbudowaniu ważonego zbioru punktów $S$ o zredukowanym rozmaiarze.
Taką technikę mogliśmy już zobaczyć w części 4.2, gdzie odpowiednio grupowaliśmy punkty w komórki kraty wykładniczej.
\\~\\
Praca \cite{DBLP:journals/ki/MunteanuS18} przedstawia inną technikę, która bazuje na budowaniu kul o wykładniczo rosnącym promieniu wokół każdego punktu z $C^{'}$.
Dla przybliżenia idei konstrukcji załóżmy, że znamy $OPT = \phi_{opt}^{k}(A)$.
Algorytm zaczyna od budowy kul o promieniach równych $\frac{1}{n}OPT$ a kończy na promieniach równych $10 OPT$, gdzie $n$ to moc zbioru $A$.
Dla takich kul budujemy $\epsilon$-pokrycie kuli.
W pracy pod pojęciem \textit{kula} rozumiemy sferę w przestrzeni o wymiarze $d$.

\begin{lemma}{\cite{pisier_1989}}
    Niech $U$ będzie sferą jednostkową w $\mathbb{R}^{d}$.
    Wtedy dla dowolnego $\epsilon \in (0,1)$, istnieje \textit{$\epsilon$-pokrycie} $B$ o rozmiarze $\Big(1 +\frac{2}{\epsilon}\Big)^{d}$, czyli dla każdego punktu $p \in U$ zachodzi:
    \begin{equation}
        \min_{b \in B} ||p-b|| \leq \epsilon
    \end{equation}
\end{lemma}

\noindent
Niestety nie istnieją efektywne metody budowania takich $\epsilon$-pokryć.
My w analizie przyjmujemy, że  $|B| = \epsilon^{-O(d)}$.
Jako referencję jak zbudować taką konstrukcję autorzy \cite{DBLP:journals/ki/MunteanuS18} sugerują analizę pracy \cite{chazelle_2000}.
Jest to problematyczne w kontekście implementacji jednak tę kwestię poruszymy w następnym rozdziale.
\begin{lemma}{\cite{DBLP:journals/ki/MunteanuS18}}
    Niech $a$, $b$, $c$ będą punktami z $\mathbb{R}^{d}$.
    Wtedy dla dowolnego $\epsilon \in (0,1)$ zachodzi:
    \begin{equation}
        \Big| ||a-c||^{2} - ||b-c||^{2} \Big| \leq \frac{12}{\epsilon} ||a-b||^2 + 2\epsilon||a-c||^2
    \end{equation}
\end{lemma}
\begin{lemma}
    Niech $A$ będzie zbiorem $n$ punktów z $\mathbb{R}^d$, $B^{i}$ będzie kulą o promieniu $r_{i} = \frac{2^{i}}{n}\sum_{x \in A} ||x||^{2}$ o środku w punkcie o współrzędnych zerowych $(0, \dots, 0)$ oraz niech $S^{i}$ będzie $\frac{\epsilon}{3}$-pokryciem kuli $B^{i}$ dla $i = 1, \dots, \log10n$.
    Zdefinujemy $S = \bigcup_{i=0}^{\log 10n} S^{i}$. 
    Wtedy:
    \begin{equation}
        \sum_{x\in A} \min_{s \in S} ||x - s||^{2} \leq \epsilon^{2} \sum_{x \in A} ||x||^{2}
    \end{equation}
\end{lemma}

\begin{proof}
    Niech $A_{close}$ będzie podzbiorem punktów z $A$, dla których kwadrat normy euklidesowej jest równy co najwyżej $\frac{1}{n}\sum_{x \in A}||x||^{2}$ oraz niech $A_{far}$ będzie zbiorem pozostałych punktów zbioru $A$.
    Ponieważ $|A_{close}| \leq n$ oraz z definicji $\epsilon$-pokrycia wynika, że:
    \begin{equation}
        \sum_{x\in A_{close}} \min_{s \in S^{0}} ||x - s||^{2} \leq |A_{close}|\frac{1}{n}\frac{\epsilon^{2}}{9}\sum_{x \in A_{close}}||x||^{2} \leq \frac{\epsilon^{2}}{9}\sum_{x \in A_{close}}||x||^{2}
    \end{equation}
    \noindent
    Dla każdego punktu $x$ ze zbioru $A_{far}$ istnieje takie $i$, że $x \in B^{i} \setminus B^{i-1}$ dla $i \in \{1, ..., \log10n \}$.
    Zatem:
    \begin{equation}
       \min_{s \in S^{i}} ||x - s||^{2} \leq \frac{\epsilon^{2}}{9} r_{i}^{2} \leq \frac{4\epsilon^{2}}{9} r_{i-1}^{2} \leq \frac{4\epsilon^{2}}{9} ||x||^2
    \end{equation}
    Sumując po wszystkich punktach otrzymujemy:
    \begin{equation}
        \sum_{x\in A} \min_{s \in S} ||x - s||^{2} \leq \frac{\epsilon^{2}}{9}\sum_{x \in A_{close}}||x||^{2} + \frac{4\epsilon^{2}}{9} \sum_{x \in A_{far}}||x||^2 < \epsilon^{2} \sum_{x \in A} ||x||^{2}
    \end{equation}
\end{proof}
\noindent
Powyżej zdefiniowana procedura zaczyna konstruckję w punkcie o współrzędnych zerowych $(0, \dots, 0)$.
W dowodzie twierdzenia 4.3 wykorzystamy lemat 4.6, aplikując go do każdego punktu z $C^{'}$.
\begin{thm}
    Dla dowolnego zbioru $n$ punktów $A \subset \mathbb{R}^d$ istnieje $(\epsilon, k)$-coreset dla problemu $k$-means zawierający $O(k\epsilon^{-d} \log n)$ punktów, gdzie $d$ to wymiar (skończony) przestrzeni.
\end{thm}
\begin{proof}
    Dla każdego z $k$ centrów ze zbioru $C^{'}$, który obliczyliśmy korzystając z algorytmu 10-aproksymacyjnego \cite{Arya2004LocalSH}, tworzymy $\log 10n$ kul o rożnych promieniach.
    Dla każdej takiej kuli o promieniu $r$ obliczamy $\frac{\epsilon}{16}r$-pokrycie.
    Niech $S$ będzie sumą wszystkich pokryć kul oraz niech $B(x)$ będzie najbliższym punktem w $S$ dla każdego punktu $x \in A$.
    Z lematu 4.6 wynika, że:
    \begin{equation}
        \sum_{x\in A} ||x - B(x)||^{2} \leq \Big( \frac{\epsilon}{16} \Big)^{2} \sum_{x\in A} ||x||^{2}
    \end{equation}
    Zauważmy, że koszt punktów $A_{c} \subseteq A$, dla których $c \in C^{'}$ jest centrem wynosi $\sum_{x\in A_{c}} ||x - c||^{2}$.
    Jeżeli założmy, że punkt $c$ jest początkiem przestrzeni to $\sum_{x\in A_{c}} ||x - c||^{2} = \sum_{x\in A_{c}} ||x||^{2}$.
    Zatem, ponieważ $C^{'}$ został uzyskany algorytmem 10-aproksymacyjnym uzyskujemy:
    \begin{equation}
        \sum_{x\in A} ||x - B(x)||^{2} \leq \Big( \frac{\epsilon}{16} \Big)^{2} \cdot 10 \cdot OPT
    \end{equation}
    Aby pokazać, że $S$ jest szukanym zbiorem rozpatrzmy dowolny zbiór $k$ centrów $C$:
    \begin{equation}
        \Big|\sum_{x\in A} \min_{c \in C} ||x - c||^{2} - \sum_{s\in S} \min_{c \in C} ||s - c||^{2}\Big|
    \end{equation}
    \begin{equation}
        \leq\Big|\sum_{x\in A} \min_{c \in C} ||x - c||^{2} - \sum_{x\in A} \min_{c \in C} ||B(x) - c||^{2}\Big|
    \end{equation}
    \begin{equation}
        \leq_{\text{lemat 4.5}} \frac{12}{\epsilon} \sum_{x\in A} ||x - B(x)||^{2} + 2\epsilon \sum_{x\in A} \min_{c \in C} ||x - c||^{2}
    \end{equation}
    \begin{equation}
        \leq \frac{12}{\epsilon} \Big( \frac{\epsilon}{16} \Big)^{2} \cdot 10 \cdot OPT + 2\epsilon \sum_{x\in A} \min_{c \in C} ||x - c||^{2}
    \end{equation}
    \begin{equation}
        \leq 2\epsilon \cdot OPT + 2\epsilon \sum_{x\in A} \min_{c \in C} ||x - c||^{2}
    \end{equation}
    \begin{equation}
        \leq 4\epsilon \sum_{x\in A} \min_{c \in C} ||x - c||^{2}
    \end{equation}
    gdzie ostatnia nierówność zachodzi ponieważ $OPT \leq \sum_{x\in A} \min_{c \in C} ||x - c||^{2}$ dla dowolnego zbioru center $C$.
    \\~\\
    Skalując $\epsilon$ przez $\frac{1}{4}$ kończymy dowód.
    Rozmiar coresetu $S$ to $O(k\epsilon^{-d} \log n)$, ponieważ obliczamy $k \log 10n$ razy $(\frac{\epsilon}{64})$- pokrycie o rozmiarze $\epsilon^{-O(d)}$.
\end{proof}

