\section{Konstrukcja}

Konstrukcja oparta jest na próbkowaniu z uwzględnieniem ważności danego punktu.
Niech $q(x)$ będzie dowolnym rozkładem prawdopodobieństwa na zbiorze $X$ oraz niech $Q \subset R^{d}$ będzie dowolnym potencjalnym zbiorem rozwiązań mocy $k$. 
Wtedy funkcję $\phi$ możemy zapisać jako:

\begin{equation}
    \phi_{X}(Q) = \sum_{x \in X} q(x) \frac{d(x, Q)^{2}}{q(x)}
\end{equation}

\noindent
Wynika z tego, że funkcja $\phi$ może być aproksymowana poprzez wylosowanie $m$ punktów z $X$ korzystając z $q(x)$ i przypisując im wagi odwrotnie proporcjonalne do $q(x)$.
Dla dowolnej liczby próbek $m$ oraz dla dowolnego rozkładu $q(x)$ możemy otrzymać sprawiedliwy (unbiased) estymator dla funkcji $\phi$.
Niestety, nie jest to wystarczające aby spełnić definicję (3.1).
W szczególności musimy zagwarantować, jednostajność wyboru dowolnego zbioru $k$ punktu $Q$ z odpowiednim prawdopodobieństwem $1 - \delta$.
Funkcja $q(x)$ może mieć wiele form, autorzy rekomendują postać:
\begin{equation}
    q(x) = \frac{1}{2}\frac{1}{|X|} + \frac{1}{2}\frac{d(x, \mu)^2}{\sum_{x^{'} \in X}d(x^{'}, \mu)^2}
\end{equation}

\begin{algorithm}
    \caption{}
\begin{algorithmic}
    \Procedure{Lightweight}{} \Comment{Require: Set of data points X, coreset size m}
        \State $\mu \leftarrow$ mean of $X$
        \For{$x \in X$}                    
            \State $q(x) = \frac{1}{2}\frac{1}{|X|} + \frac{1}{2}\frac{d(x, \mu)^2}{\sum_{x^{'} \in X}d(x^{'}, \mu)^2}$
        \EndFor
        \State $C \leftarrow$ sample $m$ weighted points from $X$ where each point $x$ has weight $\frac{1}{mq(x)}$ and is sampled with probability $q(x)$
    \EndProcedure
    \Return lightweight coreset C
\end{algorithmic}
\end{algorithm}

\noindent
Pierwszy składnik rozkładu $q(x)$ to rozkład jednostajny, który zapewnia, że każdy punkt jest wylosowany z niezerowym prawdopodobieństwem.
Drugi składnik uwzględnia kwadrat odległości punktu od średniej $\mu(X)$ dla całego zbioru.
Intuicyjnie, punkty, które są daleko od średniej $\mu(X)$ mogą mieć istotny wpływ na wartość funkcji $\phi$.
Musimy więc zapewnić, odpowiednią częstotliwość wyboru takich punktów. 
Jak pokazuje pseudokod, implementacja takiej konstrukcji jest całkiem prosta.
Zauważmy, że jest ona też bardzo praktyczna.
Algorytm przechodzi przez zbiór danych jedynie dwukrotnie, a jego złożoność to $O(nd)$.
Nie mamy zależności od $k$ co jest kluczowe w konkeście praktyczności takiego rozwiązania.