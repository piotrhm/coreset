\chapter{Lightweight Coreset}

Budowa coresetów w kontekście problemu \textit{k-means} ma bardzo długą historię.
Za przełomową pracę uznaje się \cite{Matousek99onapproximate}, która jako 
pierwsza przedstawiła w pełni wielomianowy schemat aproksymacji o złożoności $O(n\epsilon^{-2k^2d}\log^kn)$ rozwiązujący problemu $k$-means, bazując na budowie coresetu.
\\~\\
Wyróżnia się trzy techniki budowania coresetów:
\begin{itemize}
    \item Geometryczna dekompozycja problemu.
    \item Losowe próbkowanie zbioru.
    \item Zawansowane metody algebraiczne.
\end{itemize}
Pierwsza i trzecia technika cechuje się mocnymi gwarancjami teoretycznymi.
Niestety większość rozwiązań jest mało praktyczna i kosztowna czasowo.
Losowe próbkowanie w praktyce daje zadowalające wyniki jednak bardzo często nie daje nam żadnej gwarancji odnośnie optymalności rozwiązania.
Autorzy pracy \cite{bachem2017scalable} zaproponowali rozwiązanie o nazwie \textit{lightweight coreset}, które w swoich założeniach ma łączyć:
\begin{itemize}
    \item Prostą implementacje.
    \item Gwarancje teoretyczne.
    \item Szybkie działanie oparte na próbkowaniu zbioru danych.
\end{itemize}
\section{Lightweight coreset}

Zacznijmy od wprowadzenia definicji \textit{lightweight coresetu}.
\begin{definition}
    \emph{Lightweight coreset dla problemu K-means.} Niech $\epsilon > 0$ oraz $k \in \mathbb{N}$.
    Niech $X$ bedzie $n$ elementowym zbiórem punktów z $\mathbb{R}^{d}$ wraz ze średnią $\mu(X) = \frac{1}{n}\sum_{i=1}^{n} x_{i}$.
    Zbiór $C \subset \mathbb{R}^d$ jest $(\epsilon, k)$ lightweight coresetem jeżeli dla dowolnego zbioru $Q \subset \mathbb{R}^{d}$ o mocy co najwyżej $k$ zachodzi:
    \begin{equation}
        |\phi_{X}(Q) - \phi_{C}(Q)| \leq \frac{\epsilon}{2}\phi_{X}(Q) + \frac{\epsilon}{2}\phi_{X}(\mu(X))
    \end{equation}
\end{definition}

\noindent
Jak możemy zauważyć definicja (3.1) trochę się różni od (2.8).
Notacje \textit{lightweight} coresetu możemy interpretować jako relaksację gwarancji teoretycznych zdefiniowanych w (2.8).
Wprowadza ona oprócz błędu multiplikatywnego, błąd addytywny.
Nieformalnie, składnik $\frac{\epsilon}{2}\phi_{X}(Q)$ pozwala na odpowiednie skalowanie błędu aproksymacji dla funkcji $\phi$, jest on tożsamy z błędem aproksymacji coresetu zdefiniowanego w 2.8.
Druga część $\frac{\epsilon}{2}\phi_{X}(\mu(X))$ odpowiada za skalowalność rozwiązania zgodnie ze zmianą wariancji na danych.
Przez skalowalność rozumiemy zmianę rozmiaru danych, mocy zbiorów.

Głowną motywacją stojącą za konstrukcjami coresetów jest to, żeby rozwiązanie obliczone na tym zbiorze było konkurencyjne z rozwiazaniem optymalnym dla całego zbioru danych.
Dlatego w konkeście \textit{lightweight} udowodnimy następujące twierdzenie.

\begin{thm}{\cite{bachem2017scalable}}
    Niech $\epsilon \in (0, 1]$. Niech $X$ będzie skończonym zbiorem danych z $\mathbb{R}^d$ oraz niech $C$ będzie $(\epsilon, k)$ lightweight coresetem dla $X$.
    Optymalne rozwiązanie problemu K-means dla $X$ oznaczamy $Q_{X}^{*}$.
    Optymalne rozwiązanie problemu K-means dla $C$ oznaczamy $Q_{C}^{*}$.
    Dla takich założeń zachodzi:
    \begin{equation}
        \phi_{X}(Q_{C}^{*}) \leq \phi_{X}(Q_{X}^{*}) + 4\epsilon\phi_{X}(\mu(X))
    \end{equation}
\end{thm}

\begin{proof}
    Zgodnie z własnością lightweight coresetu otrzymujemy:
    \begin{equation}
        \phi_{C}(Q_{X}^{*})  \leq (1+\frac{\epsilon}{2})\phi_{X}(Q_{X}^{*}) + \frac{\epsilon}{2}\phi_{X}(\mu(X))
    \end{equation}
    oraz
    \begin{equation}
        \phi_{C}(Q_{C}^{*})  \geq (1-\frac{\epsilon}{2})\phi_{X}(Q_{C}^{*}) - \frac{\epsilon}{2}\phi_{X}(\mu(X))
    \end{equation}
    Wiemy z definicji, że $\phi_{C}(Q_{C}^{*}) \leq  \phi_{C}(Q_{X}^{*})$ oraz $1 - \frac{\epsilon}{2} \geq \frac{1}{2}$.
    A więc:
    \begin{equation}
        \phi_{X}(Q_{C}^{*}) \leq \frac{1+\frac{\epsilon}{2}}{1-\frac{\epsilon}{2}}\phi_{X}(Q_{X}^{*}) + \frac{\epsilon}{1-\frac{\epsilon}{2}}\phi_{X}(\mu(X))
    \end{equation}
    \begin{equation}
        \leq (1+2\epsilon)\phi_{X}(Q_{X}^{*}) + 2\epsilon\phi_{X}(\mu(X))
    \end{equation}
    Zauważając, że:
    \begin{equation}
        \phi_{X}(Q_{X}^{*}) \leq \phi_{X}(\mu(X))
    \end{equation}
    dowodzimy tezę twierdzenia.
\end{proof}

\noindent
Twierdzenie 1 dowodzi, że kiedy wartość $\epsilon$ maleje koszt optymalnego rozwiązania otrzymanego na zbiorze $C$ zbiega do kosztu rozwiązania otrzymanego na całym zbiorze danych.
\section{Konstrukcja}

Konstrukcja oparta jest na próbkowaniu z uwzględnieniem ważności danego punktu.
Niech $q(x)$ będzie dowolnym rozkładem prawdopodobieństwa na zbiorze $X$ oraz niech $Q \subset R^{d}$ będzie dowolnym potencjalnym zbiorem rozwiązań mocy $k$. 
Wtedy funkcję $\phi$ możemy zapisać jako:

\begin{equation}
    \phi_{X}(Q) = \sum_{x \in X} q(x) \frac{d(x, Q)^{2}}{q(x)}
\end{equation}

\noindent
Wynika z tego, że funkcja $\phi$ może być aproksymowana poprzez wylosowanie $m$ punktów z $X$ korzystając z $q(x)$ i przypisując im wagi odwrotnie proporcjonalne do $q(x)$.
Dla dowolnej liczby próbek $m$ oraz dla dowolnego rozkładu $q(x)$ możemy otrzymać sprawiedliwy (unbiased) estymator dla funkcji $\phi$.
Niestety, nie jest to wystarczające aby spełnić definicję (3.1).
W szczególności musimy zagwarantować, jednostajność wyboru dowolnego zbioru $k$ punktu $Q$ z odpowiednim prawdopodobieństwem $1 - \delta$.
Funkcja $q(x)$ może mieć wiele form, autorzy rekomendują postać:
\begin{equation}
    q(x) = \frac{1}{2}\frac{1}{|X|} + \frac{1}{2}\frac{d(x, \mu)^2}{\sum_{x^{'} \in X}d(x^{'}, \mu)^2}
\end{equation}

\begin{algorithm}
    \caption{}
\begin{algorithmic}
    \Procedure{Lightweight}{} \Comment{Require: Set of data points X, coreset size m}
        \State $\mu \leftarrow$ mean of $X$
        \For{$x \in X$}                    
            \State $q(x) = \frac{1}{2}\frac{1}{|X|} + \frac{1}{2}\frac{d(x, \mu)^2}{\sum_{x^{'} \in X}d(x^{'}, \mu)^2}$
        \EndFor
        \State $C \leftarrow$ sample $m$ weighted points from $X$ where each point $x$ has weight $\frac{1}{mq(x)}$ and is sampled with probability $q(x)$
    \EndProcedure
    \Return lightweight coreset C
\end{algorithmic}
\end{algorithm}

\noindent
Pierwszy składnik rozkładu $q(x)$ to rozkład jednostajny, który zapewnia, że każdy punkt jest wylosowany z niezerowym prawdopodobieństwem.
Drugi składnik uwzględnia kwadrat odległości punktu od średniej $\mu(X)$ dla całego zbioru.
Intuicyjnie, punkty, które są daleko od średniej $\mu(X)$ mogą mieć istotny wpływ na wartość funkcji $\phi$.
Musimy więc zapewnić, odpowiednią częstotliwość wyboru takich punktów. 
Jak pokazuje pseudokod, implementacja takiej konstrukcji jest całkiem prosta.
Zauważmy, że jest ona też bardzo praktyczna.
Algorytm przechodzi przez zbiór danych jedynie dwukrotnie, a jego złożoność to $O(nd)$.
Nie mamy zależności od $k$ co jest kluczowe w konkeście praktyczności takiego rozwiązania.
\section{Analiza}

W tej części udowodnimy, że zaproponowany w poprzedniej części algorytm oblicza lightweight coreset dla odpowiedniego $m$.

\begin{thm}{\cite{bachem2017scalable}}
    Niech $\epsilon > 0$, $\delta > 0$ oraz $k \in \mathbb{N}$. 
    Niech $X$ to skończony zbiór punktów z $\mathbb{R}^{d}$ oraz $C \subset X$ to zbiór zwracany przez algorytm dla:
    \begin{equation}
        m \geq c\frac{dk \log{k} + \log{\frac{1}{\delta}}}{\epsilon^2} 
    \end{equation}
    gdzie $c$ to stała. 
    Wtedy z prawdopodobieństwem co najmniej $1-\delta$ zbiór $C$ jest $(\epsilon, k)$ lightweight coresetem dla $X$.
\end{thm}

\begin{proof}

\noindent
Zacznę od ograniczenia ważności dla każdego punktu $x \in X$. 
W tym celu definiuje funkcję:

\begin{equation}
    f(Q) = \frac{1}{2|X|}\phi_{X}(Q) + \frac{1}{2|X|}\phi_{X}(\mu(X))
\end{equation}

\noindent
gdzie $\mu(X)$ to średnia zbioru $X$ oraz dowodzę następujący lemat. 

\begin{lemma}{\cite{bachem2017scalable}}
    Niech $X$ to skończony zbiór punktów z $\mathbb{R}^{d}$ wraz ze średnią $\mu(X)$. 
    Dla każdego $x \in X$ oraz $Q \subset \mathbb{R}^{d}$ zachodzi:
    \begin{equation}
        \frac{d(x, Q)^2}{f(Q)} \leq \frac{16d(x, \mu(X))^2}{\frac{1}{|X|}\sum_{x^{'} \in X}d(x^{'}, \mu(X))^2} + 16
    \end{equation}
\end{lemma}

\begin{proof}
    \noindent
    Z nierówności trójkąta oraz z faktu, że $(|a| + |b|)^2 = 2a^2 + 2b^2$, otrzymujemy
    \begin{equation}
        d(\mu(X), Q)^2 \leq 2d(x, \mu(X))^2 + 2d(x, Q)
    \end{equation}
    
    \noindent
    Uśrednienie dla wszystkich $x \in X$, implikuje:

    \begin{equation}
        d(\mu(X), Q)^2 \leq \frac{2}{|X|} \sum_{x \in X} d(x, \mu(X))^2 + \frac{2}{|X|} \sum_{x \in X} d(x, Q)
    \end{equation}
    \begin{equation}
       = \frac{2}{|X|} \phi_{X}(\mu(X))+ \frac{2}{|X|} \phi_{X}(Q)
    \end{equation}

    \noindent
    To implikuje, że dla każdego $x \in X$ oraz $Q \subset \mathbb{R^d}$ zachodzi:

    \begin{equation}
        d(x, Q)^2 \leq 2d(x, \mu(X))^2 + 2d(\mu(X), Q)
    \end{equation}

    \begin{equation}
       \leq 2d(x, \mu(x))^2 +  \frac{4}{|X|} \phi_{X}(\mu(X))+ \frac{4}{|X|} \phi_{X}(Q)
    \end{equation}

    \noindent
    Dzieląc powyższą nierówność przez wyżej zdefiniowaną funkcję $f(Q)$ dostajemy:

    \begin{equation}
        \frac{d(x, Q)^2}{f(Q)} \leq \frac{2d(x, \mu(x))^2 +  \frac{4}{|X|} \phi_{X}(\mu(X))+ \frac{4}{|X|} \phi_{X}(Q)}{\frac{1}{2|X|}\phi_{X}(Q) + \frac{1}{2|X|}\phi_{X}(\mu(X))}
    \end{equation}

    \begin{equation}
        \leq \frac{2d(x, \mu(x))^2 +  \frac{4}{|X|} \phi_{X}(\mu(X))}{\frac{1}{2|X|}\phi_{X}(\mu(X))} + \frac{\frac{4}{|X|} \phi_{X}(Q)}{\frac{1}{2|X|}\phi_{X}(Q)}
    \end{equation}

    \begin{equation}
        \leq \frac{16d(x, \mu(X))^2}{\frac{1}{|X|}\sum_{x^{'} \in X}d(x^{'}, \mu(X))^2} + 16
    \end{equation}

    \noindent
    co kończy dowód lematu.
\end{proof}

\noindent
Powyższy lemat implikuje, że stosunek pomiędzy kosztem kontrybucji $d(x, Q)^2$ jednego punku $x \in X$ a $f(Q)$ jest ograniczony dla każdego $Q \subset X$ przez:

\begin{equation}
    s(x) = \frac{16d(x, \mu(X))^2}{\frac{1}{|X|}\sum_{x^{'} \in X}d(x^{'}, \mu(X))^2} + 16    
\end{equation}

\noindent
Zdefinuje $S = \frac{1}{|X|} \sum_{x \in X}s(x)$ zauważając, że 
\begin{equation}
    S =  \frac{1}{|X|}  \sum_{x \in X}s(x)  =  \frac{1}{|X|} \sum_{x \in X} \Big( \frac{16d(x, \mu(X))^2}{\frac{1}{|X|}\sum_{x^{'} \in X}d(x^{'}, \mu(X))^2} + 16 \Big)
\end{equation}
\begin{equation}
    =  \frac{1}{|X|} \sum_{x \in X} \Big( \frac{16d(x, \mu(X))^2}{\frac{1}{|X|}\sum_{x^{'} \in X}d(x^{'}, \mu(X))^2} \Big) + \frac{1}{|X|} \sum_{x \in X} 16 
\end{equation}
\begin{equation}
    = \frac{16  \sum_{x \in X}  d(x, \mu(X))^2}{\sum_{x^{'} \in X}d(x^{'}, \mu(X))^2} + 16 = 32
\end{equation}
dla każdego zbioru $X$.
Dzięki temu mogę zapisać rozkład $q$ jako:

\begin{equation}
    q(x) = \frac{1}{2}\frac{1}{|X|} + \frac{1}{2}\frac{d(x, \mu(X))^2}{\sum_{x^{'} \in X}d(x^{'}, \mu(X))^2} = \frac{s(x)}{S|X|}
\end{equation}

\noindent
dla każdego $x \in X$. Rozpatruję funkcję:

\begin{equation}
    g_{Q}(x) = \frac{d(x, Q)^2}{f(Q)s(x)}
\end{equation}

\noindent
dla każdego $x \in X$ oraz $Q \subset \mathbb{R^d}$.
Zauważmy, że dla dowolnego zbioru  $Q \subset \mathbb{R}^{d}$ zachodzi:

\begin{equation}
    \phi_{X}(Q) = \sum_{x \in X} d(x, Q)^2 = S|X|f(Q) \sum_{x \in X} \frac{s(x)}{S|X|} \frac{d(x, Q)^2}{f(Q)s(x)}
\end{equation}

\begin{equation}
    =  S|X|f(Q) \sum_{x \in X} q(x) g_{Q}(x)
\end{equation}

\noindent
Następnie podstawiamy z definicji wartości oczekiwanej:

\begin{equation}
    \mathbb{E}_q[g_{Q}(x)] = \sum _{x \in X} q(x) g_{Q}(x)
\end{equation}

\noindent
dzięki temu przekształcamy ostatnie równanie:

\begin{equation}
    \phi_{X}(Q) = S|X|f(Q)\mathbb{E}_q[g_{Q}(x)]
\end{equation}

\noindent
Następnym krokiem jest ograniczenie wartości $\mathbb{E}_q[g_{Q}(x)]$.
Autorzy \cite{bachem2017scalable} nie dowodzą wprost tego ograniczenia, powołując się na inne prace.
Dowód jest bardzo skompilowany i wykracza tematyką istotnie poza ramy tej pracy więc go pomijamy.
Korzystam z finalnego ograniczenia:

\begin{equation}
    |\mathbb{E}_q[g_{Q}(x)] - \frac{1}{|C|} \sum_{x \in X}g_{X}(x)| \leq \frac{\epsilon}{32}
\end{equation}

\noindent
Powyższe ograniczenie jest prawdziwe z prawdopodobieństwem $1 - \delta$ dla dowolnego $Q \subset \mathbb{R}^{d}$ o rozmiarze nie większym niż $k$.
Mnożąc obie strony nierówności przez $32|X|f(Q)$ otrzymujemy:

\begin{equation}
    |32|X|f(Q)\mathbb{E}_q[g_{Q}(x)] - \frac{32|X|f(Q)}{|C|} \sum_{x \in X}g_{X}(x)| \leq \epsilon|X|f(Q)
\end{equation}

\noindent
Niech $(C, u)$ będzie ważonym zbiorem, gdzie dla każdego $x \in C$ definujemy funkcję $u(x) = \frac{1}{|C|q(x)}$.
Wynika z tego, że:

\begin{equation}
    \frac{32|X|f(Q)}{|C|} \sum_{x \in X}g_{X}(x) = \sum \frac{1}{|C|q(x)} d(x, Q)^2
\end{equation}

\begin{equation}
    = \sum u(x) d(x, Q)^2 = \phi_{C}(Q)
\end{equation}

\noindent
A więc otrzymujemy:

\begin{equation}
    |32|X|f(Q)\mathbb{E}_q[g_{Q}(x)] - \phi_{C}(Q)| \leq \epsilon|X|f(Q)
\end{equation}

\begin{equation}
    |\phi_{Q}(Q) - \phi_{C}(Q)| \leq \frac{\epsilon}{2}\phi_{X}(Q) + \frac{\epsilon}{2}\phi_{X}(\mu(X))
\end{equation}

\noindent
co kończy dowód twierdzenia 3.2.

\end{proof}