\section{Algorytm Gonzalez'a}

Pierwszy algorytm, który opiszemy to \textit{Farthest point algorithm} z pracy \cite{Gonzalez1985ClusteringTM}.
Jest to pierwszy algorytm aproksymacyjny rozwiązujący problem k-centrów ze współczynnikiem aproksymacji równym 2.
Jego złożoność to $O(nk)$, gdzie $n$ jest liczbą punktów danych na wejściu.
Algorytm Gonzaleza jest nam potrzebny, ponieważ w rozdziale 4.2, w którym budujemy kratę wykładniczą istotnie korzystamy z rozwiązania problemu $k$-centrów.
\\~\\
Na potrzeby tego rozdziału wprowadzimy kilka definicji.
\begin{definition}
    \emph{Problem k-centrów.} Niech $X$ będzie skończonym zbiórem punktów z $\mathbb{R}^{d}$. 
    Dla danego $X$ chcemy znaleźć zbiór $k \in \mathbb{N}$ punktów $Q \subset \mathbb{R}^{d}$, który minimalizuje funkcję $\rho_{X}(Q)$ zdefiniowaną następująco:
    \begin{equation}
        \rho_{X}(Q) = \max_{x \in X} \min_{q \in Q} \| x - q \|
    \end{equation}
\end{definition}

\begin{definition}
    Podział zbioru punktów na $k \in \mathbb{N}$ zbiorów $B_{1},\dots,B_{k}$, nazywamy \emph{k-podziałem}.
\end{definition}

\begin{definition}
    Zbiory $B_{i}$ k-podziału nazywamy \emph{klastrami}.
    W każdym klastrze wyrózniamy jeden punkt nazywając go \emph{środkiem}.
\end{definition}

\noindent
Niech $X \subset \mathbb{R}^{d}$ będzie zbiorem, na którym chcemy rozwiązać problem $k$-centrów.
Zakładamy, że $|X| > k$ ponieważ w przeciwnym przypadku problem $k$-centrów jest trywialnie rozwiązywalny.
Niech $\rho_{opt}^{k}(X)$ oznacza optymalne rozwiązanie problemu $k$-centrów dla zbioru $X$ oraz niech $T^{*}$ będzie zbiorem $k$ elementowym realizującym $\rho_{opt}^{k}(X)$.
\\~\\
Algorytm Gonzaleza rozwiązujący problem $k$-centrów składa się z fazy inicjalizującej oraz $k-1$ faz powiększających.
W fazie inicjalizującej wszystkie elementy zbioru $X$ są przypisana do zbioru $B_{1}$, który jest pierwszym klastrem.
Jeden z elementów tego zbioru oznaczamy jako $(t_{1})$ - środek klastra $B_{1}$.
Wybór tego elementu jest losowy.
Podczas $j$ fazy powiększającej, niektóre elementy z istniejącego podziału na klastry $B_{1}, \dots, B_{j}$ trafiają do nowego zbioru $B_{j+1}$.
Dodatkowo jeden z elementów nowego zbioru będzie oznaczony jako $(t_{j+1})$ - środek klastra $B_{j+1}$.
Budowę zbioru $B_{j+1}$ rozpoczynamy od wyboru punktu $v$, który należy do jednego ze zbiorów $B_{1}, \dots, B_{j}$ oraz jego odległość do środka klastra, do którego należy jest największa spośród wszystkich punktów z $B_{1}, \dots, B_{j}$. 
Taki punkt będzie oznaczony jako $(t_{j+1})$, czyli jest środkiem klastra $B_{j+1}$.
Każdy punkt, dla którego dystans do $v$ jest nie większy niż dystans do środka klastru, w którym się znajduje zostaje przeniesiony do $B_{j+1}$.
\begin{algorithm}
    \floatname{algorithm}{Algorytm}
    \caption{}
\begin{algorithmic}
    \State Algorytm na wejściu otrzymuje zbiór $X \in \mathbb{R}^{d}$.
    \State Niech $T$ będzie szukanym zbiorem $k$-centrów.
    \State Dla każdego punktu $p \notin T$, algorytm trzyma $neighbor(p)$, czyli najbliższy punkt w $T$ dla $p$ oraz $dist(p)$, czyli odległość od punktu $p$ do $neighbor(p)$.
    \Procedure{Farthest point algorithm}{}
        \State $T \leftarrow \emptyset$
        \State $dist(p) \leftarrow \infty$ for all $p \in X$
        \While{$|T| \leq k$}                    
            \State $D \leftarrow max\{dist(p) | p \in X-T\}$
            \State wybierz $v$ z $X-T$ tak, aby $dist(v) = D$
            \State add $v$ to T
            \State zaktualizuj $neighbor(p)$ oraz $dist(p)$ dla każdego $p \in X-T$
        \EndWhile
    \EndProcedure
    \Return $T$
\end{algorithmic}
\end{algorithm}
\\~\\
Algorytm 2 buduje jakiś $k$-podział oraz zbiór środków klastrów $T$.
Teraz pokażemy, że dla takiego $k$-podziału wartość funkcji celu $\rho_{X}(T)$ jest ograniczona przez $2 \cdot \rho_{X}(T^{*})$.
\\~\\
Niech $L=\rho_{X}(T)$ oraz niech $x_{0} \in T$ będzie punktem, dla którego wartość funkcji $\rho_{X}(T)$ osiągnęła maksimum.
Z konstrukcji $k$-podziału wiemy, że zbiór $T\cup \{x_{0}\}$ zawiera $k+1$ punktów, gdzie każdy z nich jest odległy od siebie o co najmniej $L$.
Zauważmy, że przynajmniej dla 2 punktów ze zbioru $T\cup \{x_{0}\}$, ich najkrótsza odległość do zbioru punktów $T^{*}$ będzie zrealizowana przez ten sam punkt $t \in T^{*}$.
Zatem przynajmniej 2 punkty odległe od siebie o co najmniej $L$ są przyporządkowane do jednego punktu z $T^{*}$.
Z nierówności trójkąta wynika, że $\rho_{X}(T^{*}) \geq \frac{L}{2}$, co z kolei implikuje, że $2\rho_{X}(T^{*}) \geq \rho_{X}(T)$.