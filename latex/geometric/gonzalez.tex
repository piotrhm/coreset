\section{Algorytm Gonzalez'a}

Pierwszy algorytm, który opiszemy to \textit{Farthest point algorithm} z pracy \cite{Gonzalez1985ClusteringTM}.
Jest to pierwszy algorytm aproksymacyjny rozwiązująych problem k-centrów ze współczynnikiem aproksymacji równym 2.
Jego złożoność to $O(nk)$, gdzie $n$ jest liczbą punktów danych na wejściu.
Algorytm Gonzaleza jest nam potrzebny, ponieważ w rozdziale 4.2, w którym budujemy kratę wykładniczą istotnie korzystamy z rozwiązania problemu $k$-centrów.
\\~\\
Na potrzeby tego rozdziału wprowadzimy kilka definicji.
\begin{definition}
    \emph{Problem k-centrów.} Niech $X$ będzie skończonym zbiórem punktów z $\mathbb{R}^{d}$. 
    Dla danego $X$ chcemy znaleźć zbiór $k \in \mathbb{N}$ punktów $Q \subset \mathbb{R}^{d}$, który minimalizuje funkcję $\rho_{X}(Q)$ zdefiniowaną następująco:
    \begin{equation}
        \rho_{X}(Q) = \max_{x \in X} \min_{q \in Q} | x - q |
    \end{equation}
\end{definition}

\begin{definition}
    Niech \emph{G = (V, E, W)} będzie ważonym nieskierowanym grafem ze zbiorem wierzchołków $V$, krawędzi $E$ oraz funkcją $W: E \rightarrow \mathbb{R}^{+}$ przyporządkowującą wagi krawędziom. 
    W tym rozdziale utożsamiamy wagi z dystansem pomiędzy dwoma punktami. 
\end{definition}

\begin{definition}
    Podział zbioru wierzchołków na $k \in \mathbb{N}$ zbiorów $B_{1},\dots,B_{k}$, nazywamy \emph{k-splitem}.
\end{definition}

\begin{definition}
    Zbiory $B_{i}$ podziału k-split nazywamy \emph{klastrami}.
\end{definition}

\noindent
Dla każdego k-splitu defninujemy funkcję celu $f: B_{1},\dots,B_{k} \rightarrow \mathbb{R}_{\geq0}$.
W tym rozdziale zakładamy, że funkcją celu jest $max(M_{1},\dots,M_{k})$, gdzie $M_{i}$ to największa waga krawędzi pomiędzy dowolnymi dwoma punktami z $B_{i}$.
Tak zdefiniowana funkcja celu jest odpowiednikiem funkcji $\rho$ zdefiniowanej dla problemu $k$-centerów.
\begin{definition}
    \emph{Problem klastrowania.} Dla danego grafu $G$, funkcji celu $f$ oraz $k \in \mathbb{N}$ znaleźć $k$-split, dla którego funkcja $f$ jest zminimalizowana.
    Dla przykładu: znaleźć k-split $(B_{1}^{*},\dots,B_{k}^{*})$ taki, że 
    \begin{equation}
        f(B_{1}^{*},\dots,B_{k}^{*}) = min\{ f(B_{1},\dots,B_{k}) | (B_{1},\dots,B_{k}) \text{ to \textit{k}-split dla \textit{G} } \}
    \end{equation}
\end{definition}

\noindent
Niech $S \subset \mathbb{R}^{d}$ będzie zbiorem, na którym chcemy rozwiązać problem $k$-centrów oraz niech $T$ będzie podzbiorem $S$.
Zakładamy, że $|S| > k$ ponieważ w przeciwnym przypadku problem podziału jest trywialnie rozwiązywalny.
\begin{definition}
Zbiór $T$ nazywamy $(k+1)$ kliką wysokości $h$ jeżeli moc zbioru $T$ jest równa $k+1$ oraz odległość pomiędzy parą dwóch rożnych punktów jest równa co najmniej $h$. 
\end{definition}

\noindent
Niech $OPT(S)$ oznacza optymalne rozwiązanie problemu k-centrów dla zbioru $S$.
Udowodnimy teraz następujący lemat, który jest potrzebny w dowodzie ograniczającym współczynnik aproksymacji algorytmu.

\begin{lemma}\cite{Gonzalez1985ClusteringTM}
    Jeżeli w zbiorze $S$ istnieje $(k+1)$ klika wysokości $h$, to $OPT(S) \geq h$.
\end{lemma}

\begin{proof}
    Kilka ma $k+1$ elementów, zatem co najmniej $2$ z nich znajdą się w jednym klastrze.
    Waga krawędzi pomiędzy tymi punktami to co najmniej $h$ co implikuje, że $OPT(S) \geq h$.
\end{proof}

\noindent
Algorytm Gonzaleza składa się z fazy inicjalizującej oraz $k-1$ faz powiększających.
W fazie inicjalizującej wszystkie elementy są przypisana do zbioru $B_{1}$, który jest pierwszym klastrem.
Jeden z elementów tego zbioru oznaczamy jako $(t_{1})$ - środek klastra $B_{1}$.
Wybór tego elementu jest losowy.
Podczas $j$ fazy powiększającej, niektóre elementy z istniejącego podziału na klastry $B_{1}, \dots, B_{j}$ trafiają do nowego zbioru $B_{j+1}$.
Dodatkowo jeden z elemntów nowego zbioru będzie oznaczony jako $(t_{j+1})$ - środek klastra $B_{j+1}$.
Budowę zbioru $B_{j+1}$ rozpoczynamy od wyboru punktu $v$, który należy do jednego ze zbiorów $B_{1}, \dots, B_{j}$ oraz jego odległość do środka klastra, do którego należy jest największa spośród wszystkich punktów z $B_{1}, \dots, B_{j}$. 
Taki punkt będzie oznaczony jako $(t_{j+1})$, czyli jest środkiem klastra $B_{j+1}$.
Każdy punkt, dla którego dystans do $v$ jest nie większy niż dystans do środka klastru, w którym się znajduje zostaje przeniesiony do $B_{j+1}$.
Algorytm 2 buduje jakiś $k$-split.
Teraz pokażemy, że dla takiego k-splitu wartość funkcji celu $f(S)$ jest ograniczona przez $2 \cdot OPT(S)$.
\begin{algorithm}
    \floatname{algorithm}{Algorytm}
    \caption{}
\begin{algorithmic}
    \State Algorytm na wejściu otrzymuje zbiór $S \in \mathbb{R}^{d}$.
    \State Niech $T$ będzie szukanym zbiorem $k$-centrów.
    \State Dla każdego punktu $p \notin T$, algorytm trzyma $neighbor(p)$, czyli najbliższy punkt w $T$ dla $p$ oraz $dist(p)$, czyli odległość od punktu $p$ do $neighbor(p)$.
    \Procedure{Farthest point algorithm}{}
        \State $T \leftarrow \emptyset$
        \State $dist(p) \leftarrow \infty$ for all $p \in S$
        \While{$|T| \leq k$}                    
            \State $D \leftarrow max\{dist(p) | p \in S-T\}$
            \State wybierz $v$ z $S-T$ tak, aby $dist(v) = D$
            \State add $v$ to T
            \State zaktualizuj $neighbor(p)$ oraz $dist(p)$ dla każdego $p \in S-T$
        \EndWhile
    \EndProcedure
    \Return $T$
\end{algorithmic}
\end{algorithm}
\\~\\
Niech $v \in B_{j}$ oraz $1 \leq j \leq k$ będzie wierzchołkiem, którego odległość do $(t_{j})$ jest maksymalna.
Niech $h$ będzie tą odległością.
Ponieważ dla zbioru wag krawędzi zachodzi nierówność trójkąta to:
\begin{equation}
    W(E(x, y)) \leq W(E(x, t_{i})) + W(E(y, t_{i})) \leq 2h
\end{equation}
gdzie $t_{i}$ to środek klastra dla punktów $x, y \in B_{i}$.
A więc możemy ograniczyć wartość naszej funkcji celu przez $f(S) \leq 2h$.
Ponieważ $v$ nigdy nie zostało wybrane na środek klastra to wiemy, że $W(t_{i}, t_{j}) \geq h$ dla $i \neq j$.
Niech $T = \{ t_{1}, \dots, t_{k}, v \}$.
Z definicji wiemy, że $T$ jest $(k+1)$ kliką wagi $h$, więc z lematu 2, $OPT(S) \geq h$.
A więc ograniczenie wartości funkcji celu to $ f(S) \leq 2h \leq 2 \cdot OPT(S)$.