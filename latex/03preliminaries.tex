\chapter{Notacja i niezbędne definicje}\label{preliminaries}
\subsection{K-means}

Zacznijmy od zdefiniowania problemu dla którego będziemy analizować konstrukcje coresetów.

\begin{definition}
    \emph{Problem K-means.} Niech $X$ to zbiór punktów z $\mathbb{R}^{d}$. 
    Dla danego $X$ chcemy znalźć zbiór $k$ punktów $Q$, który minimalizuje funkcję $\phi_{X}(Q)$ zdefiniowaną następująco:
    \begin{equation}
        \phi_{X}(Q) = \sum_{x \in X} d(x, Q)^{2} = \sum_{x \in X} \min_{q \in Q} || x - q ||_{2}^{2} 
    \end{equation}
\end{definition}

\noindent
Powyższa definicja zakłada, że działamy w przestrzeni euklidesowej.
Uogólnioną wersję można zdefiniować analogicznie, zamieniając $d$ na odpowiednią funkcję miary w danej przestrzeni.

\begin{definition}
    \emph{Problem K-means - wersja ważona.} Niech $C$ to zbiór punktów z $\mathbb{R}^{d}$ oraz niech $w$ będzie funkcją $C \rightarrow \mathbb{R}_{\ge0}$. 
    Dla danego $X$ oraz funkcji $w$ chcemy znalźć zbiór $k$ punktów $Q$, który minimalizuje funkcję $\phi_{X}(Q)$ zdefiniowaną następująco:
    \begin{equation}
        \phi_{X}(Q) = \sum_{x \in X} w(x) d(x, Q)^{2}
    \end{equation}
\end{definition}

\noindent
Ewaluację funkcji $\phi$ dla optymalnego rozwiązania oznaczamy $\phi_{OPT}^{k}(X)$. 

\subsection{Coreset}

To jak definujemy coreset ściśle zależy od problemu, który optymalizujemy.
Zacznijmy od podstawowej definicji coresetu dla problemu \textit{K-means}.

\begin{definition}
    \emph{Coreset.} Niech $X$ to zbiór punktów z $\mathbb{R}^{d}$ oraz niech $Q$ to dowolny podzbiór $X$ rozmiaru co najwyżej $k$. 
    Zbiór $C$ nazywamy $(\epsilon, k)$ coresetem jeżeli zachodzi:
    \begin{equation}
        |\phi_{X}(Q) - \phi_{C}(Q)| \leq \epsilon\phi_{X}(Q)
    \end{equation}
\end{definition}

\noindent
Zauważmy, że taka definicja daje nam bardzo moce gwarancje teoretyczne.
Ewaluacji coresetu $\phi_{C}(Q)$ musi aproksymować $\phi_{X}(Q)$ z dokładnością $1+\epsilon$ jednocześnie dla dowolnego zbioru $k$ kandydatów na rozwiązanie $Q$ w kontekście całego zbioru $X$.
Jest to na tyle istotne, że w literaturze odróżnia się tą wersję nazywając ją \textit{Strong Coreset}.

\noindent