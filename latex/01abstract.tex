\thispagestyle{empty}
\begin{abstract}
    W pracy przedstawimy stan wiedzy na temat budowania coresetów w kontekście problemu $k$-means.
    W szczególności omówimy techniki konstrukcji coresetów takie jak geometryczna dekompozycja oraz losowe próbkowanie z \cite{DBLP:journals/ki/MunteanuS18}.

    Celem pracy jest implementacja technik budowania coresetów oraz przeprowadzenie eksperymentów, które zbadają jakość uzyskanych coresetów \cite{DBLP:journals/ki/MunteanuS18} \cite{bachem2017scalable}.
    W tym celu porównamy wyniki działania algorytmu Lloyd'a rozwiązującego problem $k$-means na całym zbiorze danych z jego działaniem na uzyskanych różnymi metodami coresetach.
\end{abstract}
