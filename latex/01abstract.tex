\thispagestyle{empty}
\begin{abstract}
    W pracy przedstawię stan wiedzy na temat budowania coresetów w kontekście algorytmu K-means.
    W szczególności poruszę techniki takie jak geometryczna dekompozycja oraz losowe próbkowanie bazując na badaniach \cite{DBLP:journals/ki/MunteanuS18}.

    Celem pracy jest przedstawienie wyników teoretycznych \cite{DBLP:journals/ki/MunteanuS18} jak i implementacja technik budowania coresetów, które mogą mieć zastosowanie praktyczne \cite{bachem2017scalable} \cite{10.1145/1007352.1007400}.
    Następnie porównam ich działanie testując je na zbiorach punktów z dwuwymiarowej przestrzeni euklidesowej.
\end{abstract}
