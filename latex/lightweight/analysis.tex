\section{Analiza}

W tym podrozdziale pokażemy, że zaproponowany w poprzedniej części algorytm oblicza lightweight coreset dla odpowiedniego $m$.

\begin{thm}{\cite{bachem2017scalable}}
    Niech $\epsilon > 0$, $\delta > 0$ oraz $k \in \mathbb{N}$. 
    Niech $X$ będzie skończonym zbiór punktów z $\mathbb{R}^{d}$ oraz $C \subseteq X$ to zbiór zwracany przez algorytm dla:
    \begin{equation}
        m \geq c\frac{dk \log{k} + \log{\frac{1}{\delta}}}{\epsilon^2} 
    \end{equation}
    gdzie $c$ jest stałą. 
    Wtedy z prawdopodobieństwem co najmniej $1-\delta$ zbiór $C$ jest $(\epsilon, k)$ lightweight coresetem dla $X$.
\end{thm}

\begin{proof}

\noindent
Zaczniemy od ograniczenia $q(x)$ dla każdego punktu $x \in X$. 
W tym celu definiuje funkcję:

\begin{equation}
    f(Q) = \frac{1}{2|X|}\phi_{X}(Q) + \frac{1}{2|X|}\phi_{X}(\mu(X))
\end{equation}

\noindent
gdzie $\mu(X)$ to średnia zbioru $X$ oraz dowodnimy następujący lemat. 

\begin{lemma}{\cite{bachem2017scalable}}
    Niech $X$ będzie skończonym zbiórem punktów z $\mathbb{R}^{d}$ wraz ze średnią $\mu(X)$. 
    Dla każdego $x \in X$ oraz $Q \subset \mathbb{R}^{d}$ zachodzi:
    \begin{equation}
        \frac{d(x, Q)^2}{f(Q)} \leq \frac{16d(x, \mu(X))^2}{\frac{1}{|X|}\sum_{x^{'} \in X}d(x^{'}, \mu(X))^2} + 16
    \end{equation}
\end{lemma}

\begin{proof}
    \noindent
    Z nierówności trójkąta oraz z faktu, że $(|a| + |b|)^2 = 2a^2 + 2b^2$, otrzymujemy
    \begin{equation}
        d(\mu(X), Q)^2 \leq 2d(x, \mu(X))^2 + 2d(x, Q)
    \end{equation}
    
    \noindent
    Uśrednienie dla wszystkich $x \in X$, implikuje:

    \begin{equation}
        d(\mu(X), Q)^2 \leq \frac{2}{|X|} \sum_{x \in X} d(x, \mu(X))^2 + \frac{2}{|X|} \sum_{x \in X} d(x, Q)
    \end{equation}
    \begin{equation}
       = \frac{2}{|X|} \phi_{X}(\mu(X))+ \frac{2}{|X|} \phi_{X}(Q)
    \end{equation}

    \noindent
    To implikuje, że dla każdego $x \in X$ oraz $Q \subset \mathbb{R^d}$ zachodzi:

    \begin{equation}
        d(x, Q)^2 \leq 2d(x, \mu(X))^2 + 2d(\mu(X), Q)
    \end{equation}

    \begin{equation}
       \leq 2d(x, \mu(x))^2 +  \frac{4}{|X|} \phi_{X}(\mu(X))+ \frac{4}{|X|} \phi_{X}(Q)
    \end{equation}

    \noindent
    Dzieląc powyższą nierówność przez wyżej zdefiniowaną funkcję $f(Q)$ dostajemy:

    \begin{equation}
        \frac{d(x, Q)^2}{f(Q)} \leq \frac{2d(x, \mu(x))^2 +  \frac{4}{|X|} \phi_{X}(\mu(X))+ \frac{4}{|X|} \phi_{X}(Q)}{\frac{1}{2|X|}\phi_{X}(Q) + \frac{1}{2|X|}\phi_{X}(\mu(X))}
    \end{equation}

    \begin{equation}
        \leq \frac{2d(x, \mu(x))^2 +  \frac{4}{|X|} \phi_{X}(\mu(X))}{\frac{1}{2|X|}\phi_{X}(\mu(X))} + \frac{\frac{4}{|X|} \phi_{X}(Q)}{\frac{1}{2|X|}\phi_{X}(Q)}
    \end{equation}

    \begin{equation}
        \leq \frac{16d(x, \mu(X))^2}{\frac{1}{|X|}\sum_{x^{'} \in X}d(x^{'}, \mu(X))^2} + 16
    \end{equation}

    \noindent
    co kończy dowód lematu.
\end{proof}

\noindent
Powyższy lemat implikuje, że stosunek pomiędzy kosztem kontrybucji $d(x, Q)^2$ jednego punku $x \in X$ a $f(Q)$ jest ograniczony dla każdego $Q \subseteq X$ przez:

\begin{equation}
    s(x) = \frac{16d(x, \mu(X))^2}{\frac{1}{|X|}\sum_{x^{'} \in X}d(x^{'}, \mu(X))^2} + 16    
\end{equation}

\noindent
Niech $S = \frac{1}{|X|} \sum_{x \in X}s(x)$.
Zauważmy, że:
\begin{equation}
    S =  \frac{1}{|X|}  \sum_{x \in X}s(x)  =  \frac{1}{|X|} \sum_{x \in X} \Big( \frac{16d(x, \mu(X))^2}{\frac{1}{|X|}\sum_{x^{'} \in X}d(x^{'}, \mu(X))^2} + 16 \Big)
\end{equation}
\begin{equation}
    =  \frac{1}{|X|} \sum_{x \in X} \Big( \frac{16d(x, \mu(X))^2}{\frac{1}{|X|}\sum_{x^{'} \in X}d(x^{'}, \mu(X))^2} \Big) + \frac{1}{|X|} \sum_{x \in X} 16 
\end{equation}
\begin{equation}
    = \frac{16  \sum_{x \in X}  d(x, \mu(X))^2}{\sum_{x^{'} \in X}d(x^{'}, \mu(X))^2} + 16 = 32
\end{equation}
dla każdego zbioru $X$.
Dzięki temu możemy zapisać rozkład $q$ jako:

\begin{equation}
    q(x) = \frac{1}{2}\frac{1}{|X|} + \frac{1}{2}\frac{d(x, \mu(X))^2}{\sum_{x^{'} \in X}d(x^{'}, \mu(X))^2} = \frac{s(x)}{S|X|}
\end{equation}

\noindent
dla każdego $x \in X$. Teraz zdefinujmy funkcję:

\begin{equation}
    g_{Q}(x) = \frac{d(x, Q)^2}{f(Q)s(x)}
\end{equation}

\noindent
dla każdego $x \in X$ oraz $Q \subset \mathbb{R}^{d}$.
Zauważmy, że dla dowolnego zbioru  $Q \subset \mathbb{R}^{d}$ zachodzi:

\begin{equation}
    \phi_{X}(Q) = \sum_{x \in X} d(x, Q)^2 = S|X|f(Q) \sum_{x \in X} \frac{s(x)}{S|X|} \frac{d(x, Q)^2}{f(Q)s(x)}
\end{equation}

\begin{equation}
    =  S|X|f(Q) \sum_{x \in X} q(x) g_{Q}(x)
\end{equation}

\noindent
Następnie podstawiamy z definicji wartości oczekiwanej:

\begin{equation}
    \mathbb{E}_q[g_{Q}(x)] = \sum _{x \in X} q(x) g_{Q}(x)
\end{equation}

\noindent
dzięki temu przekształcamy ostatnie równanie:

\begin{equation}
    \phi_{X}(Q) = S|X|f(Q)\mathbb{E}_q[g_{Q}(x)]
\end{equation}

\noindent
Następnym krokiem jest ograniczenie wartości $\mathbb{E}_q[g_{Q}(x)]$.
Autorzy \cite{bachem2017scalable} nie dowodzą wprost tego ograniczenia, powołując się na inne prace \cite{LI2001516}.
Dowód jest bardzo skompilowany i wykracza tematyką istotnie poza ramy tej pracy, więc go pomijamy.
Korzystamy z finalnego ograniczenia:

\begin{equation}
    |\mathbb{E}_q[g_{Q}(x)] - \frac{1}{|C|} \sum_{x \in X}g_{X}(x)| \leq \frac{\epsilon}{32}
\end{equation}

\noindent
Powyższe ograniczenie jest prawdziwe z prawdopodobieństwem $1 - \delta$ dla dowolnego $Q \subset \mathbb{R}^{d}$ o rozmiarze nie większym niż $k$.
Mnożąc obie strony nierówności przez $32|X|f(Q)$ otrzymujemy:

\begin{equation}
    |32|X|f(Q)\mathbb{E}_q[g_{Q}(x)] - \frac{32|X|f(Q)}{|C|} \sum_{x \in X}g_{X}(x)| \leq \epsilon|X|f(Q)
\end{equation}

\noindent
Niech $(C, u)$ będzie ważonym zbiorem, gdzie dla każdego $x \in C$ definujemy funkcję $u(x) = \frac{1}{|C|q(x)}$.
Wynika z tego, że:

\begin{equation}
    \frac{32|X|f(Q)}{|C|} \sum_{x \in X}g_{X}(x) = \sum \frac{1}{|C|q(x)} d(x, Q)^2
\end{equation}

\begin{equation}
    = \sum u(x) d(x, Q)^2 = \phi_{C}(Q)
\end{equation}

\noindent
A więc otrzymujemy:

\begin{equation}
    |32|X|f(Q)\mathbb{E}_q[g_{Q}(x)] - \phi_{C}(Q)| \leq \epsilon|X|f(Q)
\end{equation}

\begin{equation}
    |\phi_{Q}(Q) - \phi_{C}(Q)| \leq \frac{\epsilon}{2}\phi_{X}(Q) + \frac{\epsilon}{2}\phi_{X}(\mu(X))
\end{equation}

\noindent
co kończy dowód twierdzenia 3.2.

\end{proof}